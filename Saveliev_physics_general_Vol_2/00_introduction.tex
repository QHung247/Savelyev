% !TEX root = saveliev_physics_general_course_2.tex
%!TEX TS-program = pdflatex
%!TEX encoding = UTF-8 Unicode


\chapter*{INTRODUCTION}\label{chap:chapter_introduction}
\addcontentsline{toc}{chapter}{Introduction}
\chaptermark{\sc Introduction}

Physics is a science dealing with the most general properties and forms of motion of matter.

A classical definition of matter was given by V.~Lenin in his book \textit{Materialism and Empirio-Criticism}: ``Matter is a philosophical category denoting the objective reality which is given to man by his sensations, and which is copied, photographed and reflected by our sensations, while existing independently of the''\footnote{V.~I.~Lenin. \textit{Collected Works}, Vol. 14, p. 130. Moscow, Foreign Languages Publishing House (1962).}. Two propositions are significant in this definition, namely, (1) matter is what exists objectively, \ie, independently of anyone's consciousness or sensations, and (2) matter is copied and reflected by our sensations and, consequently, is cognizable.

It follows from the definition of physics that it concentrates knowledge accumulated on the most general properties and phenomena of the world surrounding us. As academician S.~Vavilov noted in one of his articles, ``the extremely common character of a considerable part of the contents of physics, its facts and laws drew physics and philosophy together from time immemorial\ldots. Sometimes physical statements have such a nature that they are difficult to distinguish and separate from philosophical statements, and a physicist must be a philosopher''.

Two kinds of matter are known at present: substance and field. The first kind of matter---substance---includes, for example, atoms, molecules, and all bodies built of them. Electromagnetic, gravitational, and other fields form the second kind of matter. The different kinds of matter can change into each other. For instance, an electron and a positron (representatives of substance) may transform into photons (\ie, into an electromagnetic field). The reverse process is also possible.

Matter is in continuous motion, which is understood to mean any change in general in dialectical materialism\footnote{Dialectical materialism is the name given to the Marxist-Leninist philosophy. The fundamental issue of any philosophy as to what is primary---matter or consciousness---is solved by dialectic materialism in favour of matter when it states that matter is primary and consciousness is secondary. The method of this philosophy is dialectics. It considers matter in constant motion and development whose source is contained in the internal contradictions inherent in objects and phenomena themselves.}. Motion is an inalienable property of matter, which, like matter itself, cannot be created or destroyed. Matter exists and moves in space and in time, which are forms of existence of matter.

The laws of physics are established by generalizing experimental facts. They express the objective regularities existing in nature. These laws are customarily expressed in the form of quantitative relationships between various physical quantities.

The fundamental method of investigation in physics is the running of an experiment, \ie, the observation of the phenomenon being studied in accurately controlled conditions. The latter must permit one to watch the course of the phenomenon and reproduce it each time when these conditions are repeated. Phenomena can be produced experimentally that are not observed in nature. For example, more than ten of the chemical elements known at present have meanwhile not been discovered in nature and were obtained artificially by means of nuclear reactions.

Hypotheses are enlisted to explain experimental data. A hypothesis is a scientific assumption advanced to explain a definite fact or phenomenon and requiring verification and proving to become a scientific theory or law. The correctness of a hypothesis is verified by running the corresponding experiments and by determining whether the corollaries following from the hypothesis agree with the results of experiments and observations. A hypothesis that has successfully passed such verification and has been proved becomes a scientific law or theory.

A physical theory is a system of basic ideas summarizing experimental data and reflecting the objective regularities of nature. A physical theory explains a whole field of natural phenomena from a single viewpoint.

Physics is subdivided into the so-called classical physics and quantum physics. The term classical is applied to the physics whose creation was completed at the beginning of the 20th century. Classical physics was founded by Isaac Newton (1642-1727), who formulated the fundamental laws of classical mechanics. Newtonian mechanics proved to be exceedingly fruitful and mighty, and physicists acquired the conviction that any physical phenomenon can be explained with the aid of Newton's laws.

The edifice of classical physics built up by the end of the last century was very harmonious. Most physicists were convinced that they already knew everything about nature that could be known. The most perspicacious physicists, however, understood that the edifice of classical physics had weak spots. For example, the British physicist William Thomson (Lord Kelvin, 1824-1907) said that there are two dark clouds on the horizon of the cloudless sky of classical physics---the unsuccessful attempts to set up a theory of blackbody radiation, and the contradictory behaviour of ether---the hypothetical medium in which light waves were supposed to propagate. The persistent attempts to surmount these difficulties led to unexpected results. To solve these problems, which were beyond the possibilities of classical physics, it became necessary to revise quite radically the established, habitual notions and introduce concepts that were alien to the spirit of classical physics. Max Planck (1858-1947) succeeded in solving the problem of blackbody radiation in 1900 by introducing the concept of light emission in separate portions---quanta. Thus, at the threshold of the 20th century, the concept of the quantum appeared. It plays an exceedingly important part in modern physics and has resulted in the creation of quantum mechanics.

The contradictory nature of the experimental facts relating to ether induced Albert Einstein (1879-1955) to revise the notions of space and time that were considered to be obvious from Newton's times. The result was the appearance of the theory of relativity. The latter gives equations of motion appreciably differing from those of Newtonian mechanics for bodies travelling with speeds that are noticeable in comparison with the speed of light.

The year 1897 saw the discovery of the electron. The atoms of all the chemical elements were found to contain these particles. Thus, atoms, previously considered indivisible, appeared to have a complicated structure.

The beginning of the 20th century was thus marked in physics by the radical breaking down of numerous habitual concepts and notions. New physical discoveries and theories destroyed the notions of the structure of matter formed by many physicists. Some of them interpreted this as the vanishing of matter. Many physicists lapsed into idealism, and a crisis began in physics.

V.~Lenin in his book \textit{Materialism and Empirio-Criticism} written in 1908 gave annihilating criticism of ``physical'' idealism. He showed that the new discoveries indicate not the vanishing of matter, but the vanishing of the limit up to which matter was known before that time. ``Matter disappears'', wrote Lenin, ``means that the limit within which we have hitherto known matter disappears and that our knowledge is penetrating deeper; properties of matter are likewise disappearing which formerly seemed absolute, immutable, and primary (impenetrability, inertia, mass, etc.) and which are now revealed to be relative and characteristic only of certain states of matter. For the sole 'property' of matter with whose recognition philosophical materialism is bound up is the property of being an objective reality, of existing outside the mind.''\footnote{V.~I.~Lenin. \textit{Collected Works}, Vol. 14, p. 260. Moscow, Foreign Languages Publishing House (1962).}.

The process of recognizing the world is infinite. Our knowledge at any given stage of development of science is due to the historically achieved level of cognition and cannot be considered as final or complete. It is of necessity relative knowledge, \ie, requires further development, further verification, and more precise definition. At the same time, any truly scientific theory, notwithstanding its relativity and incompleteness, contains elements of absolute, \ie, complete, knowledge, and thus signifies a step in the cognition of the objective world. For instance, mechanics based on Newton's laws is not correct, strictly speaking. But for a certain range of phenomena, this mechanics is quite satisfactory. Thus, the development of science did not cross out Newtonian mechanics. It only established the limits within which it is correct. Newtonian mechanics formed a constituent part of the general edifice of the physical science.

The beginning of the 20th century is characterized by persistent attempts to penetrate into the internal structure of atoms. The key to determining their structure was found to be the studying of atomic spectra. The theory of the atom developed by Niels Bohr (1885-1962) in 1913 was the first striking success in explaining the observed spectra. This theory, however, has obvious features of inconsistency: in addition to the motion of an electron in an atom obeying the laws of classical mechanics, the theory imposes special quantum restrictions on this motion. The theory soon had to pay for this lack of consistency. After the first successes in explaining the spectra of the simplest atom---that of hydrogen---it was found that Bohr's theory is unable to explain the behaviour of atoms with two or more electrons.

The need to develop a new comprehensive theory of atoms became pressing. A bold hypothesis of Louis de Broglie put forward in 1924 placed the cornerstone in such a theory. It was known by that time that light, while being a wave process, also exhibits a corpuscular nature in a number of cases, \ie, behaves like a stream of particles. De Broglie put forth the idea that the particles of a substance, in turn, should display wave properties too in definite conditions. De Broglie's hypothesis soon received a brilliant experimental confirmation---it was proved that a wave process is associated with the particles of a substance, and it must be taken into account when considering the mechanics of an atom. A result of this discovery was the development by Erwin Schr\"odinger (1887-1961) and Werner Heisenberg (1901-1976) of a new physical theory---wave or quantum mechanics. The latter achieved striking successes in explaining atomic processes and the structure of a substance. Results were obtained that showed excellent agreement with experimental data when 'it was found possible to surmount the mathematical difficulties.

The latest decades were noted by remarkable achievements in the field of studying the atomic nucleus. Scientists and engineers have mastered nuclear processes to such an extent that the practical use of nuclear energy has become possible. One of the leading places in this field belongs to Soviet physics. Particularly, the first atomic power plant in the world was erected in the USSR.

Finally, in recent years, the walls of laboratories created by the hands of man were moved apart beyond the limits of our globe. On October 4, 1957, an artificial satellite of the Earth was launched in the Soviet Union the first time in history. It was a small laboratory outfitted with apparatus for scientific research. April 12, 1961, saw the first flight of a man into outer space. The first Soviet cosmonaut, Yuri Gagarin, flew around the Earth and landed safely. The first space rockets were built in the Soviet Union. They left the field of the Earth's attraction and transmitted to the Earth by means of radio signals valuable results of studying outer space and, particularly, photographs of the reverse side of the Moon. In 1969, U.S. astronauts landed on the Moon. In 1975, two Soviet automatic spaceships made a soft landing on Venus and transmitted valuable information on the physical conditions on this planet, and also photographs of its surface.

There is no doubt that the nearest future will be marked with new fundamental discoveries in the science of physics.

%\pagestyle{mystyle}
