% !TEX root = saveliev_physics_general_course_2.tex
%!TEX TS-program = pdflatex
%!TEX encoding = UTF-8 Unicode

\appendix

\chapter*{APPENDICES}\label{chap:A}
\addcontentsline{toc}{chapter}{APPENDICES}
\chaptermark{APPENDICES}
\renewcommand{\thechapter}{A}
\setcounter{section}{0}
\renewcommand{\thesection}{\thechapter.\arabic{section}}
% \setcounter{theorem}{0}
% \renewcommand{\thetheorem}{\thechapter.\arabic{theorem}}
\setcounter{equation}{0}
\renewcommand{\theequation}{\thechapter.\arabic{equation}}
\setcounter{figure}{0}
\setcounter{table}{0}
\renewcommand{\thetable}{\thechapter.\arabic{table}}

\section{List of Symbols}\label{sec:A_1}

\begin{table}[h]
	% \renewcommand{\arraystretch}{1.2}
	% \caption{ }
	% \vspace{-0.6cm}
	% \label{table:12_1}
	\begin{center}\resizebox{\linewidth}{!}{
        \begin{threeparttable}[b]
			\begin{tabular}{p{0.5cm} p{11cm}}
                $A$ & amplitude; gas amplification; work\\
				$\vec{A}$ & vector potential of magnetic field\tnote{${\dagger}$}$\,\,$\\
                $a$ &  amplitude\\
				$\vec{a}$ & acceleration; vector\\
                $B$ & Kerr constant\\
				$\vec{B}$ & magnetic induction\\
                $C$ &  capacitance; circulation of vector; Curie constant\\
                $c$ &  electromagnetic constant; speed of light\\
                $D$ &  angular dispersion\\
				$\vec{D}$ & electric displacement\\
                $d$ &  period of diffraction grating; separation distance of capacitor plates\\
                $\div$ &  divergence\\
                $E$ &  illuminance; Young's modulus\\
				$\vec{E}$ & electric field strength\\
				$\vec{E}^*$ & strength of extraneous force field\\
				$\mathcal{E}$ & electromotive force (e.m.f.)\\
                $e$ &  base of natural logarithms; positive elementary charge\\
                $\hatvec{e}$ &  unit vector\\
                $F$ &  Faraday constant\\
                $\vec{F}$ &  force\\
                $f$ &  focal length\\
                $G$ &  shear modulus\\
			\end{tabular}
            \begin{tablenotes}
    			\item [${\dagger}$] The magnitude of a vector is denoted by the same symbol as the vector itself, but in ordinary italic (sloping) type.
    		\end{tablenotes}
        \end{threeparttable}
	}\end{center}
\end{table}

\begin{table}[h]
	% \renewcommand{\arraystretch}{1.2}
	% \caption{ }
	% \vspace{-0.6cm}
	% \label{table:12_1}
	\begin{center}\resizebox{\linewidth}{!}{
			\begin{tabular}{p{0.5cm} p{11cm}}
				$\vec{H}$ & magnetic field strength\\
				$\hslash$ & Planck's constant $h$ divided by $2\pi$\\
				$I$ &  current; luminous intensity; sound intensity\\
                $i$ &  imaginary unity ($i=\sqrt{-1}$)\\
                $\vec{j}$ &  current density; density of energy flux\\
				$\vec{K}$ & momentum\\
				$k$ & constant of proportionality; wavenumber\\
				$\vec{k}$ & wavevector\\
				$L$ & inductance; loudness level; luminance; optical path\\
				$\vec{L}$ & angular momentum\\
				$l$ & length; mean free path\\
				$\vec{l}$ & displacement\\
				$M$ & luminous emittance; magnification; mass of mole\\
				$\vec{M}$ & magnetization; moment of force\\
				$m$ & mass\\
				$N$ & demagnetization factor; number\\
				$\ab{N}{A}$ & Avogadro constant\\
				$n$ & number; refractive index\\
				$P$ & degree of polarization; optical power; power; probability; radiated power\\
				$\vec{P}$ & force of gravity; polarization\\
				$p$ & pressure\\
				$\vec{p}$ & dipole moment; electric moment\\
				$Q$ & amount of heat; quality of oscillator circuit\\
				$q$ & charge\\
				$R$ & molar gas constant; radius; resistance; resolving power\\
				$\ab{R}{H}$ & Hall coefficient\\
				$\Re$ & real number\\
				$r$ & distance; radius\\
				$\vec{r}$ & position vector\\
				$S$ & area\\
				$\vec{S}$ & Poynting vector\\
				$T$ & absolute temperature; period\\
				$\vec{T}$ & torque\\
				$t$ & time\\
				$U$ & voltage\\
				$u$ & mobility of ion\\
				$\vec{u}$ & velocity\\
				$V$ & Verdet constant; visibility\\
				$\vec{v}$ & velocity\\
			\end{tabular}
	}\end{center}
\end{table}

\begin{table}[h]
	% \renewcommand{\arraystretch}{1.2}
	% \caption{ }
	% \vspace{-0.6cm}
	% \label{table:12_1}
	\begin{center}\resizebox{\linewidth}{!}{
			\begin{tabular}{p{0.5cm} p{11cm}}
				$W$ & energy\\
				$w$ & energy density\\
				$X$ & reactance\\
				$x$ & coordinate\\
				$y$ & coordinate\\
				$Z$ & atomic number of element; impedance\\
				$z$ & coordinate; valence\\
				$\alpha$ & angle; drag coefficient; initial phase of oscillations; rotational constant\\
				$\beta$ & angle; polarizability of molecule; relative velocity\\
				$\gamma$ & angle; attenuation coefficient\\
				$\Delta$ & difference in optical path; increment; Laplacian operator\\
				$\delta$ & density of metal; fraction of energy; phase difference\\
				$\varepsilon$ & relative permittivity; strain\\
				$\varepsilon_0$ & electric constant\\
				$\theta$ & angle; polar angle; polar coordinate\\
				$\varkappa$ & thermal conductivity; wave absorption coefficient\\
				$\varkappa'$ &  extinction coefficient\\
				$\lambda$ & linear charge density; logarithmic decrement; wavelength\\
				$\mu$ & permeability\\
				$\ab{\mu}{B}$ & Bohr magneton\\
				$\mu_0$ & magnetic constant\\
				$\nu$ & frequency\\
				$\xi$ & displacement of wave point\\
				$\pi$ & ratio of circumference to diameter\\
				$\rho$ & coherence radius; density; reflection coefficient; resistivity; volume density of charge\\
				$\sigma$ & conductivity; cross-sectional area; stress; surface charge density\\
				$\tau$ & retardation time; time; time constant of a circuit; transmission coefficient\\
				$\Phi$ & flux\\
				$\varphi$ & angle; azimuthal angle; potential\\
				$\chi$ & electric susceptibility\\
				$\ab{\chi}{m}$ & magnetic susceptibility\\
				$\Psi$ & flux linkage; total magnetic flux\\
				$\psi$ & angle\\
				$\Omega$ & solid angle\\
				$\omega$ & angular frequency\\
				$\vec{\omega}$ & angular velocity\\
				$\nabla$ & del (Hamiltonian) operator\\
			\end{tabular}
	}\end{center}
\end{table}

\clearpage

\section[Units of Electrical and Magnetic Quantities]{Units of Electrical and Magnetic Quantities in the International System (SI)
and in the Gaussian System}\label{sec:A_2}

The electric constant
\begin{equation*}
	\varepsilon_0 = \frac{1}{4\pi (2.997925)^2 \times \num{e9}} \si{F.m^{-1}} \approx \frac{1}{4\pi \times \num{9e9}} \si{F.m^{-1}}.
\end{equation*}

The magnetic constant
\begin{equation*}
	\mu_0 = 4\pi \times \num{e-7} \si{H.m^{-1}}.
\end{equation*}

The electromagnetic constant
\begin{equation*}
	c = \frac{1}{\sqrt{\varepsilon_0\mu_0}} = 2.997925 \times \num{e8} \si{m.s^{-1}} \approx \SI{3e8}{m.s^{-1}}.
\end{equation*}

The relations between the units are given approximately.
To obtain the exact values, substitute $2.997925$ for $3$ and $(2.997925)^2$ for $9$.

\begin{table}[b]
	\renewcommand{\arraystretch}{1.2}
	\caption{}
	\vspace{-0.6cm}
	\label{table:A_1}
	\begin{center}\resizebox{0.98\linewidth}{!}{
			\begin{tabular}{llll}
				\toprule[1pt]
				\multirow{2}{*}{\textbf{Quantity and aymbol}} & \multicolumn{2}{c}{\textbf{Unit and symbol}} & \multirow{2}{*}{\textbf{Relation}}\\
				\cline{2-3}
				& \textbf{SI} & \textbf{Gaussian} & \\
				\midrule[0.5pt]\midrule[0.5pt]
				Force $F$ & newton (N) & dyne (dyn) & $\SI{1}{N} = \SI{e5}{dyn}$\\
				Work $A$ and energy $W$ & joule (J) & erg (erg) & $\SI{1}{J}=\SI{e7}{erg}$\\
				Charge $q$ & coulomb (C) & \cgse{q} & $\SI{1}{C}= \SI{3e9}{\cgse{q}}$\\
				Electric field strength $E$ & volt per metre (\si{V.m^{-1}}) & \cgse{E} & $\SI{1}{C}=\SI{3e4}{.m^{-1}}$\\
				Potential $\varphi$, voltage $U$, e.m.f. $\mathcal{E}$ & volt (V) & \cgse{\varphi,U, \mathcal{E}} & $\SI{1}{\cgse{\varphi,U,\mathcal{E}}}=\SI{300}{V}$\\
				Electric moment of dipole $p$ & coulomb-metre (\si{C.m}) & \cgse{p} & $\SI{1}{C.m}=\SI{3e11}{\cgse{q}}$\\
				Polarization $P$ & coulomb per metre squared (\si{C.m^{-2}}) & \cgse{P} & $\SI{1}{C.m^{-2}}=\SI{3e5}{\cgse{P}}$\\
				 Electric susceptibility $\chi$ & SI$_{\chi}$ & \cgse{\chi} & $\SI{1}{\cgse{\chi}}=4\pi$ SI$_{\chi}$\\
				 Electric displacement $D$ & coulomb per metre squared (\si{C.m^{-2}}) & \cgse{D} & $\SI{1}{C.m^{-2}}=4\pi\SI{3e5}{\cgse{D}}$\\
				 Flux of electric displacement $\Phi$ & coulomb (C) & \cgse{\Phi} & $\SI{1}{C}= 4\pi\SI{3e9}{\cgse{\Phi}}$\\
				 Capacitance $C$ & farad (F) & centimetre (cm) & $\SI{1}{F}= \SI{e11}{cm}$\\
				 Current $I$ & ampere (A) & \cgse{I} & $\SI{1}{A}= \SI{3e9}{\cgse{I}}$\\
				 Current density $j$ & ampere per metre squared (\si{A.m^{-2}}) & \cgse{j} & $\SI{1}{A.m^{-2}}= \SI{3e5}{\cgse{j}}$\\
				 Resistance $R$ & ohm (\si{\ohm}) & \cgse{R} & $\SI{1}{\cgse{R}}= \SI{9e11}{\ohm}$\\
				 Resistivity $\rho$ & ohm-metre (\si{\ohm.m}) & \cgse{\rho} & $\SI{1}{\cgse{\rho}}= \SI{9e9}{\ohm.m}$\\
				 Conductivity $\sigma$ & siemens per metre (\si{S.m^{-1}}) & \cgse{\sigma} & $\SI{1}{S.m^{-1}}= \SI{9e9}{\cgse{\sigma}}$\\
				 Magnetic induction $B$ & tesla (T) & gauss (Gs) & $\SI{1}{T}=\SI{e4}{Gs}$\\
				 Flux of magnetic induction $\Phi$ & weber (Wb) & maxwell (Mx) & $\SI{1}{Wb}= \SI{e8}{Mx}$\\
				 Flux linkage $\Psi$ & weber (Wb) & maxwell (Mx) & $\SI{1}{Wb}=\SI{e8}{Mx}$\\
				 Magnetic moment $\ab{p}{m}$ & ampere-metre squared (\si{A.m^2}) & \cgs{m}{\ab{p}{m}} & $\SI{1}{A.m^2}= \SI{e3}{\cgs{m}{\ab{p}{m}}}$\\
				 Magnetization $M$ & ampere per metre (\si{A.m^{-1}}) & \cgs{m}{M} & $\SI{1}{\cgs{m}{M}}= \SI{e3}{A.m^{-1}}$\\
				 Magnetic field strength $H$ & ampere per metre (\si{A.m^{-1}}) & oersted (Oe) & $\SI{1}{A.m^{-1}}= 4\pi\SI{e-3}{Oe}$\\
				 Magnetic susceptibility $\ab{\chi}{m}$ & SI$_{\ab{\chi}{m}}$ & \cgs{m}{{\ab{\chi}{m}}} & $\SI{1}{\cgs{m}{{\ab{\chi}{m}}}}= 4\pi$ SI$_{\ab{\chi}{m}}$\\
				 Inductance $L$ & henry (H) & centimetre (cm) & $\SI{1}{H}= \SI{e9}{cm}$\\
				 Mutual inductance $L_{12}$ & henry (H) & centimetre (cm) & $\SI{1}{H}= \SI{e9}{cm}$\\
				\bottomrule[1pt]
			\end{tabular}
	}\end{center}
\end{table}

\clearpage

\section[Basic Formulas of Electricity and Magnetism]{Basic Formulas of Electricity and Magnetism in the SI and in the Gaussian System}\label{sec:A_3}
\parindent 0mm

1. Coulomb's law:
\begin{equation*}
	F = \dfrac{1}{4\pi\varepsilon_0} \dfrac{q_1q_2}{r^2} \siun F = \dfrac{q_1q_2}{r^2} \gsun.
\end{equation*}

2. Electric field strength (definition):
\begin{equation*}
	\vec{E} = \dfrac{\vec{F}}{q}.
\end{equation*}

3. Field strength of point charge:
\begin{equation*}
	E = \dfrac{1}{4\pi\varepsilon_0} \dfrac{q}{\varepsilon r^2} \siun E=\dfrac{q}{\varepsilon r^2} \gsun.
\end{equation*}

4. Field strength between charged planes and near surface of charged conductor:
\begin{equation*}
	E=\frac{\sigma}{\varepsilon_0\varepsilon} \siun E=\frac{4\pi\sigma}{\varepsilon} \gsun.
\end{equation*}

5. Potential (definition):
\begin{equation*}
	\varphi = \frac{\ab{W}{p}}{q}.
\end{equation*}

6. Potential of field of point charge:
\begin{equation*}
	\varphi=\frac{1}{4\pi\varepsilon_0} \frac{q}{\varepsilon r} \siun \varphi=\frac{q}{\varepsilon r} \gsun.
\end{equation*}

7. Work of field forces on charge:
\begin{equation*}
	A = q (\varphi_1 - \varphi_2).
\end{equation*}

8. Relation between $\vec{E}$ and $\varphi$:
\begin{equation*}
	\vec{E} = - \gradop{\varphi}.
\end{equation*}

9. Relation between $\varphi$ and $\vec{E}$:
\begin{equation*}
	\varphi_1 - \varphi_2 = \int_1^2 \vec{E}\ccdot\derivec{l}.
\end{equation*}

10. Curl of vector $\vec{E}$ for electrostatic field:
\begin{equation*}
	\curlop{\vec{E}} = 0.
\end{equation*}

11. Circulation of vector $\vec{E}$ for electrostatic field:
\begin{equation*}
	\oint \vec{E}\ccdot\derivec{l} = 0.
\end{equation*}

12. Electric moment of dipole:
\begin{equation*}
	p = ql.
\end{equation*}

13. Torque acting on dipole in electric field:
\begin{equation*}
	\vec{T} = \vecprod{p}{E}.
\end{equation*}

14. Energy of dipole in field:
\begin{equation*}
	W = - \vecprod{p}{E}.
\end{equation*}

15. Dipole moment of ``elastic'' molecule:
\begin{equation*}
	\vec{p}=\beta\varepsilon_0\vec{E} \siun  \vec{p}=\beta\vec{E} \gsun.
\end{equation*}

16. Polarization (definition):
\begin{equation*}
	\vec{P} = \frac{1}{\Delta{V}}\sum{\vec{p}}.
\end{equation*}

17. Relation between $\vec{P}$ and $\vec{E}$:
\begin{equation*}
	\vec{P}=\chi\varepsilon_0\vec{E} \siun \vec{P}=\chi\vec{E} \gsun.
\end{equation*}

18. Relation between $\vec{P}$ and volume density of bound charges:
\begin{equation*}
	\rho' = - \divop{\vec{P}}.
\end{equation*}

19. Relation between $\vec{P}$ and surface density of bound charges:
\begin{equation*}
	\sigma' = \ab{P}{n}.
\end{equation*}

20. Electric displacement (definition):
\begin{equation*}
	\vec{D} = \varepsilon_0\vec{E} + \vec{P} \siun \vec{D} = \vec{E} + 4\pi\vec{P} \gsun.
\end{equation*}

21. Divergence of vector $\vec{D}$:
\begin{equation*}
	\divop{\vec{D}}=\rho \siun \divop{\vec{D}}=4\pi\rho \gsun.
\end{equation*}

22. Gauss's theorem for $\vec{D}$:
\begin{equation*}
	\oint \vec{D}\ccdot\derivec{S}=\sum q \siun \oint \vec{D}\ccdot\derivec{S}=4\pi\sum q \gsun.
\end{equation*}

23. Relation between permittivity $\varepsilon$ and electric susceptibility $\chi$:
\begin{equation*}
	\varepsilon = 1+\chi \siun \varepsilon = 1+4\pi\chi \gsun.
\end{equation*}

24. Relation between values of $\chi$ in the SI and in the Gaussian system:
\begin{equation*}
	\ab{\chi}{SI} = 4\pi\ab{\chi}{GS}.
\end{equation*}

25. Relation between $\vec{D}$ and $\vec{E}$:
\begin{equation*}
	\vec{D}=\varepsilon\varepsilon_0\vec{E} \siun \vec{D}=\varepsilon\vec{E} \gsun.
\end{equation*}

26. Relation between $\vec{D}$ and $\vec{E}$ in
a vacuum:
\begin{equation*}
	\vec{D}=\varepsilon_0\vec{E} \siun \vec{D}=\vec{E} \gsun.
\end{equation*}

27. $\vec{D}$ of point charge field:
\begin{equation*}
	D=\frac{1}{4\pi} \frac{q}{r^2} \siun D=\frac{q}{r^2} \gsun.
\end{equation*}

28. Capacitance of capacitor (definition):
\begin{equation*}
	C = \frac{q}{U}.
\end{equation*}

29. Capacitance of plane capacitor:
\begin{equation*}
	C=\frac{\varepsilon_0\varepsilon S}{d} \siun C=\frac{\varepsilon_0\varepsilon S}{4\pi d} \gsun.
\end{equation*}

30. Energy of system of charges:
\begin{equation*}
	W = \frac{1}{2} \sum q\varphi.
\end{equation*}

31. Energy of charged capacitor:
\begin{equation*}
	W = \frac{CU^2}{2}.
\end{equation*}

32. Density of electric field energy:
\begin{equation*}
	w=\frac{\varepsilon_0\varepsilon E^2}{2} \siun w=\frac{\varepsilon E^2}{8\pi} \gsun.
\end{equation*}

33. Current (definition):
\begin{equation*}
	I = \diff{q}{t}.
\end{equation*}

34. Current density (definition):
\begin{equation*}
	j = \diff{I}{S_{\perp}}.
\end{equation*}

35. Continuity equation:
\begin{equation*}
	\divop{\vec{j}} = - \diffpartial{\rho}{t}.
\end{equation*}

36. Voltage (definition):
\begin{equation*}
	U = \varphi_1 - \varphi_2 + \mathcal{E}_{12}.
\end{equation*}

37. Ohm's law:
\begin{equation*}
	I = \frac{U}{R}.
\end{equation*}

38. Ohm's law in differential form
\begin{equation*}
	\vec{j} = \frac{1}{\rho} \vec{E} = \sigma\vec{E}.
\end{equation*}

39. Joule-Lenz law:
\begin{equation*}
	Q = \int_0^t RI^2\, \deriv{t}.
\end{equation*}

40. Joule-Lenz law in differential form:
\begin{equation*}
	w = \rho j^2.
\end{equation*}

41. Force of interaction of two parallel currents in a vacuum (per unit length):
\begin{equation*}
	F=\frac{\mu_0}{4\pi}\frac{2I_1I_2}{b} \siun F=\frac{1}{c^2}\frac{2I_1I_2}{b} \gsun.
\end{equation*}

42. Field of freely moving charge:
\begin{equation*}
	\vec{B} = \frac{\mu_0}{4\pi}\frac{q (\vecprod{v}{r})}{r^3} \siun
	\vec{B} = \frac{1}{c}\frac{q (\vecprod{v}{r})}{r^3}.
\end{equation*}

43. Biot-Savart law:
\begin{equation*}
	\derivec{B}=\frac{\mu_0}{4\pi} \frac{I (\derivec{l}\times\vec{r})}{r^3} \siun \derivec{B}=\frac{1}{c} \frac{I (\derivec{l}\times\vec{r})}{r^3} \gsun.
\end{equation*}

44. Lorentz force:
\begin{equation*}
	\vec{F}=q\vec{E} + q (\vecprod{v}{B}) \siun \vec{F}=q\vec{E} + \frac{q}{c} (\vecprod{v}{B}) \gsun.
\end{equation*}

45. Ampere's law:
\begin{equation*}
	\derivec{F}=I (\derivec{l}\times\vec{B}) \siun \derivec{F}=\frac{1}{c} I (\derivec{l}\times\vec{B}) \gsun.
\end{equation*}

46. Magnetic moment of loop with current:
\begin{equation*}
	\ab{p}{m}=IS \siun \ab{p}{m}=\frac{1}{c}IS \gsun.
\end{equation*}

47. Angular momentum exerted on magnetic moment in a magnetic field:
\begin{equation*}
	\vec{L} = \ab{\vec{p}}{m} \times \vec{B}.
\end{equation*}

48. ``Mechanical'' energy of magnetic moment in a magnetic field:
\begin{equation*}
	W = -\ab{\vec{p}}{m}\ccdot\vec{B}.
\end{equation*}

49. Divergence of vector $\vec{B}$:
\begin{equation*}
	\divop{\vec{B}} = 0.
\end{equation*}

50. Gauss's theorem for $\vec{B}$:
\begin{equation*}
	\oint\vec{B}\ccdot\derivec{S} = 0.
\end{equation*}

51. Magnetization (definition):
\begin{equation*}
	\vec{M} = \frac{1}{\Delta{V}} \sum\ab{\vec{p}}{m}.
\end{equation*}

52. Magnetic field strength (definition):
\begin{equation*}
	\vec{H}=\frac{1}{\mu_0} \vec{B} - \vec{M} \siun \vec{H}=\vec{B} - 4\pi\vec{M} \gsun.
\end{equation*}

53. Relation between $\vec{M}$ and $\vec{H}$:
\begin{equation*}
	\vec{M} = \ab{\chi}{m} \vec{H}.
\end{equation*}

54. Relation between permeability $\mu$ and magnetic susceptibility $\ab{\chi}{m}$:
\begin{equation*}
	\mu=1+\ab{\chi}{m} \siun \mu=1+4\pi\ab{\chi}{m} \gsun.
\end{equation*}

55. Relation between values of $\ab{\chi}{m}$ in the SI and in the Gaussian system:
\begin{equation*}
	\ab{\chi}{m,SI} = 4\pi\ab{\chi}{m,GS}.
\end{equation*}

56. Relation between $\vec{B}$ and $\vec{H}$:
\begin{equation*}
	\vec{B}=\mu\mu_0\vec{H} \siun \vec{B}=\mu\vec{H} \gsun.
\end{equation*}

57. Relation between $\vec{B}$ and $\vec{H}$ a vacuum:
\begin{equation*}
	\vec{B}=\mu_0\vec{H} \siun \vec{B}=\vec{H} \gsun.
\end{equation*}

58. Curl of vector $\vec{H}$ for a stationary field:
\begin{equation*}
	\curlop{\vec{H}} = \vec{j} \siun \curlop{\vec{H}} = \frac{4\pi}{c} \vec{j} \gsun.
\end{equation*}

59. Circulation of vector $\vec{H}$ for a stationary field:
\begin{equation*}
	\oint \vec{H}\ccdot\derivec{l} = \sum I \siun \oint \vec{H}\ccdot\derivec{l} = \frac{4\pi}{c}\sum I \gsun.
\end{equation*}

60. Magnetic field strength of straight current:
\begin{equation*}
	H = \frac{1}{4\pi}\frac{2I}{b} \siun H = \frac{1}{c}\frac{2I}{b} \gsun.
\end{equation*}

61. Magnetic field strength at centre of ring current:
\begin{equation*}
	H = \frac{I}{2R} \siun H = \frac{1}{c} \frac{2\pi I}{R} \gsun.
\end{equation*}

62. Field strength of solenoid:
\begin{equation*}
	H = nI \siun H = \frac{4\pi}{c} nI \gsun.
\end{equation*}

63. Flux of magnetic induction (definition):
\begin{equation*}
	\Phi = \int_S \vec{B}\ccdot\derivec{S}.
\end{equation*}

64. Work done on loop with current when it is moved in a magnetic field:
\begin{equation*}
	A = I \Delta{\Phi} \siun A = \frac{1}{c} I \Delta{\Phi} \gsun.
\end{equation*}

65. Flux linkage or total magnetic flux (definition):
\begin{equation*}
	\Psi = \sum\Phi.
\end{equation*}

66. Induced e.m.f.:
\begin{equation*}
	\ab{\mathcal{E}}{i}=-\diff{\Psi}{t} \siun \ab{\mathcal{E}}{i}=- \frac{1}{c}\diff{\Psi}{t} \gsun.
\end{equation*}

67. Inductance (definition):
\begin{equation*}
	L = \frac{\Psi}{I} \siun L = c \frac{\Psi}{I} \gsun.
\end{equation*}

68. Inductance of solenoid:
\begin{equation*}
	L = \mu_\mu n^2l S \siun L = 4\pi\mu_\mu n^2l S \gsun.
\end{equation*}

69. E.m.f. of self-induction (in absence of ferromagnetics):
\begin{equation*}
	\ab{\mathcal{E}}{s}=-L\diff{I}{t} \siun \ab{\mathcal{E}}{s}=-\frac{1}{c^2}L\diff{I}{t} \gsun.
\end{equation*}

70. Energy of magnetic fteld of current:
\begin{equation*}
	W=\frac{LI^2}{2} \siun W=\frac{1}{c^2}\frac{LI^2}{2} \gsun.
\end{equation*}

71. Density of energy of magnetic field:
\begin{equation*}
	w=\frac{\mu_0\mu H^2}{2} \siun w=\frac{\mu H^2}{8\pi} \gsun.
\end{equation*}

72. Energy of linked loops with current:
\begin{equation*}
	W = \frac{1}{2} \sum_{i,k} L_{ik} I_i I_k \siun W = \frac{1}{2c^2} \sum_{i,k} L_{ik} I_i I_k \gsun.
\end{equation*}

73. Density of displacement current:
\begin{equation*}
	\ab{\vec{j}}{dis} = \dot{\vec{D}} \siun \ab{\vec{j}}{dis} = \frac{1}{4\pi}\dot{\vec{D}} \gsun.
\end{equation*}

74. Maxwell's equations in differential form:
\begin{align*}
	\curlop{\vec{E}}&=-\diffpartial{\vec{B}}{t}\siun \curlop{\vec{E}}=-\frac{1}{c}\diffpartial{\vec{B}}{t}\siun\\
	\divop{\vec{B}}&=0\siun  \divop{\vec{B}}=0\gsun\\
	\curlop{\vec{H}}&=\vec{j}+\diffpartial{\vec{D}}{t} \siun \curlop{\vec{H}}=\frac{4\pi}{c}\vec{j}+\frac{1}{c}\diffpartial{\vec{D}}{t} \gsun\\
	\divop{\vec{D}}&=\rho \siun \divop{\vec{D}}=4\pi\rho \gsun.
\end{align*}

75. Maxwell s equations in integral form:
\begin{align*}
	\oint_{\Gamma}\vec{E}\ccdot\derivec{l}&=-\int_S\diffpartial{\vec{B}}{t}\ccdot\derivec{S} \siun \oint_{\Gamma}\vec{E}\ccdot\derivec{l}=-\frac{1}{c}\int_S\diffpartial{\vec{B}}{t}\ccdot\derivec{S} \gsun \\
	\oint_S\vec{B}\ccdot\derivec{S}&=0 \siun \oint_S\vec{B}\ccdot\derivec{S}=0 \gsun\\
	\oint_{\Gamma}\vec{H}\ccdot\derivec{l}&=\int_S\vec{j}\ccdot\derivec{S}+\int_S\diffpartial{\vec{D}}{t}\ccdot\derivec{S} \siun \\ \oint_{\Gamma}\vec{H}\ccdot\derivec{l}&=\frac{4\pi}{c}\int_S\vec{j}\ccdot\derivec{S}+\frac{1}{c}\int_S\diffpartial{\vec{D}}{t}\ccdot\derivec{S} \gsun\\
	\oint_S \vec{D}\ccdot\derivec{S}&=\int_V\rho\,\deriv{V} \siun \oint_S \vec{D}\ccdot\derivec{S}=4\pi\int_V\rho\,\deriv{V} \gsun.
\end{align*}

76. Velocity of electromagnetic waves:
\begin{equation*}
	v = \frac{c}{\sqrt{\varepsilon\mu}}.
\end{equation*}

77. Relation between amplitudes of vectors $\vec{E}$ and $\vec{H}$ in an electromagnetic wave:
\begin{equation*}
	\ab{E}{m}=\sqrt{\varepsilon_0\varepsilon}=\ab{H}{m}\sqrt{\mu_0\mu} \siun \ab{E}{m}=\sqrt{\varepsilon}=\ab{H}{m}\sqrt{\mu} \gsun.
\end{equation*}

78. Poynting vector:
\begin{equation*}
	\vec{S} = \vecprod{E}{H} \siun \vec{S} = \frac{1}{4\pi c}\vecprod{E}{H} \gsun.
\end{equation*}

79. Density of electromagnetic field momentum:
\begin{equation*}
	\vec{K}=\frac{1}{c^2}\vecprod{E}{H} \siun \vec{K}=\frac{1}{4\pi c^2}\vecprod{E}{H} \gsun.
\end{equation*}
