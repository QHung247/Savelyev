% !TEX root = saveliev_physics_general_course_2.tex
%!TEX TS-program = pdflatex
%!TEX encoding = UTF-8 Unicode


\chapter[Giao thoa ánh sáng]{Giao thoa ánh sáng}\label{chap:17}
\chaptermark{Giao thoa ánh sáng}

\section{Giao thoa sóng ánh sáng}\label{sec:17_1}

Giả sử ta có hai sóng cùng tần số, chồng chất lên nhau, tạo ra các dao động cùng phương, cụ thể là
\begin{equation*}
    A_1 \cos(\omega t + \alpha_1),\quad A_2 \cos(\omega t + \alpha_2),
\end{equation*}

\noindent
tại một điểm xác định trong không gian.
Biên độ của dao động tổng hợp tại điểm đã cho được xác định bằng biểu thức
\begin{equation*}
    A^2 = A_1^2 + A_2^2 + 2 A_1 A_2 \cos\delta,
\end{equation*}

\noindent
trong đó $\delta=\alpha_2-\alpha_1$ [xem phương trình (7.84) quyển I].

Nếu độ lệch pha $\delta$ của các dao động thiết lập bởi các sóng là không đổi theo thời gian thì các sóng đó được gọi là \textbf{kết hợp}\footnote{Chúng ta sẽ thảo luận khái niệm kết hợp chi tiết hơn ở phần sau.}.

Độ lệch pha $\delta$ của các sóng không kết hợp thay đổi liên tục và nhận bất kỳ giá trị nào với xác suất như nhau.
Do vậy, giá trị trung bình theo thời gian của $\cos\delta$ bằng không.
Vì thế,
\begin{equation*}
    \average{A^2} = \average{A_1^2} + \average{A_2^2}.
\end{equation*}

\noindent
Kết hợp với \eqn{16_10}, ta kết luận rằng cường độ quan sát được từ sự chồng chất của các sóng không kết hợp bằng với tổng của cường độ từng sóng.
% the intensity observed upon the superposition of incoherent waves equals the sum of the intensities produced by each of the waves individually:

\begin{equation}\label{eq:17_1}
    I = I_1 + I_2.
\end{equation}

% For coherent waves, $\cos\delta$ has a time-constant value (but a different one for each point of space), so that,
Đối với sóng kết hợp, $\cos\delta$ mang hằng số độc lập thời gian (nhưng khác nhau tại mỗi điểm trong không gian), vậy nên,
\begin{equation}\label{eq:17_2}
    I = I_1 + I_2 + 2 \sqrt{I_1 I_2} \cos \delta.
\end{equation}

\noindent
% At the points of space for which $\cos\delta>0$, the intensity $I$ will exceed $I_1+I_2$; at the points for which $\cos\delta<0$, it will be smaller than $I_1+I_2$. 
% Thus, the superposition of coherent light waves is attended by redistribution of the light flux in space.
% As a result, maxima of the intensity will appear at some spots and minima at others.
% This phenomenon is called the interference of waves.
% Interference manifests itself especially clearly when the intensity of both interfering waves is the same: $I_1=I_2$.
% Hence, according to \eqn{17_2}, at the maxima $I=4I_1$, while at the minima $I=0$.
% For incoherent waves in the same condition, we get the same intensity $I=2I_1$ everywhere [see \eqn{17_1}].
Tại những điểm trong không gian có $\cos\delta>0$, cường độ $I$ sẽ đạt $I_1+I_2$; tại những điểm có $\cos\delta<0$, giá trị $I$ sẽ nhỏ hơn $I_1+I_2$. 
Do đó, sự chồng chập của các sóng ánh sáng kết hợp đi kèm với sự phân bố của thông lượng ánh sáng trong không gian.
Hệ quả là, cực đại sẽ xuất hiện ở một số điểm và cực tiểu ở những điểm còn lại.
Giao thoa biểu hiện rất rõ ràng khi cường độ của 2 sóng giao thoa bằng nhau: $I_1=I_2$.
Do đó, dựa vào \eqn{17_2}, tại cực đại $I=4I_1$ thì cực tiểu là $I=0$.
Đối với sóng không kết hợp ở cùng một điều kiện, ta nhận được cường độ giống nhau $I=2I_1$ tại mọi điểm [xem \eqn{17_1}].

% It follows from what has been said above that when a surface is illuminated by several sources of light (for example, by two lamps), an interference pattern ought to be observed with a characteristic alternation of maxima and minima of intensity.
% We know from our everyday experience, however, that in this case the illumination of the surface diminishes monotonously with an increasing distance from the light sources, and no interference pattern is observed.
% The explanation is that natural light sources are not coherent.
Từ những điều đã nói ở trên, có thể suy ra rằng khi mặt phẳng được chiếu sáng bởi nhiều nguồn sáng (ví dụ, bởi 2 đèn), ta có thể quan sát được các vân sáng, tối xen kẽ.
Tuy nhiên, trong đời sống, cường độ ánh sáng giảm dần đều khi khoảng cách giữa nguồn sáng và điểm đang xét tăng, và các vân giao thoa sẽ không xuất hiện.
Nguyên nhân cho điều này là do nguồn sáng tự nhiên không thể kết hợp.

% The incoherence of natural light sources is due to the fact that the radiation of a luminous body consists of the waves emitted by many atoms.
% The individual atoms emit wave trains with a duration of about \SI{e-8}{\second} and a length of about \SI{3}{\metre} (see \sect{16_1}).
% The phase of a new train is not related in any way to that of the preceding one.
% In the light wave emitted by a body, the radiation of one group of atoms after about \SI{e-8}{\second} is replaced by the radiation of another group, and the phase of the resultant wave undergoes random changes.
Sự không kết hợp của nguồn sáng tự nhiên là do bức xạ của vật bao gồm nhiều sóng bị phát xạ bởi nhiều phân tử.
Các phân tử riêng lẻ phát ra chuỗi sóng có thời lượng khoảng \SI{e-8}{\second} và bước sóng khoảng \SI{3}{\metre} (xem \sect{16_1}).
Pha của chuỗi sóng mới không liên quan gì đến chuỗi sóng trước.
Trong bức xạ của một vật, bức xạ của một nhóm các phân tử sau \SI{e-8}{\second} được thay thế bởi bức xạ của nhóm khác, và pha của sóng tổng hợp này trải qua những thay đổi ngẫu nhiên.

% Coherent light waves can be obtained by splitting (by means of reflections or refractions) the wave emitted by a single source into two parts.
% If these waves are made to cover different optical paths and are then superposed onto each other, interference is observed.
% The difference between the optical paths covered by the interfering waves must not be very great because the oscillations being added must belong to the same resultant wave train.
% If this difference will be of the order of one metre, oscillations corresponding to different trains will be superposed, and the phase difference between them will continuously change in a chaotic way.
Có thể đạt được sóng ánh sáng kết hợp bằng cách tách (bằng phương pháp phản xạ hoặc tán xạ) sóng ánh sáng từ một nguồn thành hai phần.
Nếu như các sóng này được điều hướng tới các quang trình khác nhau thì có thể quan sát được sự giao thoa.
Sự khác biệt giữa các quang trình không được quá lớn bởi vì dao động được thêm vào phải thuộc cùng một chuỗi sóng tổng hợp.
Nếu độ biến thiên ở mức một mét, dao động tương ứng với mỗi sóng sẽ bị chồng chập, và độ lệch pha giữa chúng sẽ thay đổi hỗn loạn.

\begin{figure}[!htb]
	\begin{center}
		\includegraphics[scale=1]{figures/ch_17/fig_17_1.pdf}
		\caption[]{}
        % \caption[]{Splitting into two coherent waves at point $0$. One wave travels the path $s_1$ in a medium of refractive index $n_1$, and the second wave travels the path $s_2$, in a medium of refractive index $n_2$.}
		\label{fig:17_1}
	\end{center}
	\vspace{-0.8cm}
\end{figure}

% Assume that the splitting into two coherent waves occurs at point $0$ (\fig{17_1}).
% Up to point P, the first wave travels the path $s_1$ in a medium of refractive index $n_1$, and the second wave travels the path $s_2$, in a medium of refractive index $n_2$.
% If the phase of oscillations at point $0$ is $\omega t$, then the first wave will produce the oscillation $A_1\cos\omega(t-s_1/v_1)$ at point P, and the second wave, the oscillation $A_2\cos\omega(t-s_2/v_2)$ at this point; $v_1=c/n_1$ and $v_2=c/n_2$ are the phase velocities of the waves.
% Hence, the difference between the phases of the oscillations produced by the waves at point P will be
Cho rằng 2 sóng kết hợp được tách từ điểm $0$ (\fig{17_1}).
Ta xét điểm P, sóng đầu tiên truyền theo đường $s_1$ trong không gian có chiết xuất $n_1$, và sóng thứ hai truyền theo đường $s_2$ trong không gian có chiết xuất $n_2$.
Nếu pha của dao động tại điểm $0$ là $\omega t$, thì sóng đầu tiên sẽ có dao động $A_1\cos\omega(t-s_1/v_1)$ tại điểm P, và sóng thứ hai có dao động $A_2\cos\omega(t-s_2/v_2)$ tại điểm đề cập; $v_1=c/n_1$ và $v_2=c/n_2$ là hai vận tốc pha của các sóng.
Do đó, độ lệch pha của dao động tạo bởi các sóng tại điểm P sẽ là
\begin{equation*}
    \delta = \omega \parenthesis{\frac{s_2}{v_2} - \frac{s_1}{v_1}} = \frac{\omega}{c} (n_2 s_2 - n_1 s_1).
\end{equation*}

\noindent
% Replacing $\omega/c$ with $2\pi\nu/c = 2\pi/\lambda_0$ (where $\lambda_0$ is the wavelength in
% a vacuum), the expression for the phase difference can be written in the form
Thay thế $\omega/c$ với $2\pi\nu/c = 2\pi/\lambda_0$ (với $\lambda_0$ là bước sóng trong chân không), biểu thức về độ lệch pha sẽ là
\begin{equation}\label{eq:17_3}
    \delta = \frac{2 \pi}{\lambda_0} \Delta,
\end{equation}

\noindent
%where
với
\begin{equation}\label{eq:17_4}
    \Delta = n_2 s_2 - n_1 s_1 = L_1 - L_2,
\end{equation}

\noindent
% is a quantity equal to the difference between the optical paths travelled by the waves and is called the \textbf{difference in optical path} [compare with \eqn{16_55}].
là đại lượng bằng với độ lệch khoảng cách giữa các quang trình của các sóng và được gọi là \textbf{độ lệch quang trình} [so với \eqn{16_55}]

% A glance at \eqn{17_3} shows that if the difference in the optical path equals an integral number of wavelengths in a vacuum:
Phương trình \eqn{17_3} cho biết rằng nếu độ lệch quang trình bằng số nguyên lần bước sóng trong chân không:
\begin{equation}\label{eq:17_5}
    \Delta = \pm m \lambda_0 \quad (m = 0, 1, 2, \ldots),
\end{equation}

\noindent
% then the phase difference $\delta$ is a multiple of $2\pi$, and the oscillations produced at point P by both waves will occur with the same phase.
% Thus, \eqn{17_5} is the condition for an interference maximum, \ie, for \textbf{constructive interference}.
thì độ lệch pha $\delta$ là bội số của $2\pi$, và các dao động tại điểm P cùng pha.
Do đó, \eqn{17_5} là điều kiện để giao thoa đạt cực đại, \ie, đối với \textbf{giao thoa biên độ cực đại}.

% If $\Delta$ equals a half-integral number of wavelengths in a vacuum:
Nếu $\Delta$ bằng với nửa số nguyên lần bước sóng trong chân không:
\begin{equation}\label{eq:17_6}
    \Delta = \pm \parenthesis{m + \frac{1}{2}} \lambda_0 \quad (m = 0, 1, 2, \ldots),
\end{equation}

\noindent
% then, $\delta=\pm(2m + 1)\pi$, so that the oscillations at point P are in counterphase.
% Thus, \eqn{17_6} is the condition for an interference minimum, \ie, for \textbf{destructive interference}.
thì, $\delta=\pm(2m + 1)\pi$, do đó các dao động tại điểm P ngược pha nhau.
Vậy nên, \eqn{17_6} là điều kiện để giao thoa đạt cực tiểu, \ie, đối với \textbf{giao thoa biên độ cực tiểu}.

% Let us consider two cylindrical coherent light waves emerging from sources S$_1$ and S$_2$ having the form of parallel thin luminous filaments or narrow slits (\fig{17_2}).
% The region in which these waves overlap is called the \textbf{interference field}.
% Within this entire region, there are observed alternating places with maximum and minimum intensity of light.
% If we introduce a screen into the interference field, we shall see on it an interference pattern having the form of alternating light and dark fringes.
% Let us calculate the width of these fringes, assuming that the screen is parallel to a plane passing through sources S$_1$ and S$_2$.
% We shall characterize the position of a point on the screen by the coordinate x measured in a direction at right angles to lines S$_1$ and S$_2$.
% We shall choose the beginning of our readings at point $0$ relative to which S$_1$ and S$_2$ are arranged symmetrically.
% We shall consider that the sources oscillate in the same phase.
% Examination of \fig{17_2} shows that
Xét 2 sóng ánh sáng kết hợp hình trụ phát ra từ điểm $S_1$ và $S_2$ có dạng sợi mỏng song song nhau hoặc qua khe hẹp (\fig{17_2}).
Nơi mà hai sóng chồng chập lên nhau được gọi là \textbf{vùng giao thoa}.
Trong vùng này, có thể quan sát được các nơi có cường độ cực đại và cực tiểu xen kẽ nhau.
Nếu ta thêm một màn hứng vào bên trong vùng giao thoa, ta có thể quan sát được các vân giao thoa có dạng vân sáng và tối xen kẽ.
Tính toán chiều rộng của các vân, cho rằng màn hứng song song với mặt phẳng cắt 2 điểm $S_1$ và $S_2$.
Ta có thể xác định vị trí của một điểm trên màn bằng tọa độ x được đo theo hướng vuông góc với các đường thẳng $S_1$ và $S_2$.
Ta chọn điểm gốc $0$ là giao điểm của đường trung bình của $S_1S_2$ với màn.
Ta lưu ý rằng các nguồn phải có cùng pha.
Từ \fig{17_2}, ta có
\begin{equation*}
    s_1^2 = l^2 + \parenthesis{x - \frac{d}{2}}^2,\quad s_2^2 = l^2 + \parenthesis{x + \frac{d}{2}}^2.
\end{equation*}

\begin{figure}[!htb]
	\begin{center}
		\includegraphics[scale=0.95]{figures/ch_17/fig_17_2.pdf}
		\caption[]{}
        % \caption[]{Interference field of two cylindrical coherent light waves emerging from sources S$_1$ and S$_2$ having the form of parallel thin luminous filaments or narrow slits.}
		\label{fig:17_2}
	\end{center}
	\vspace{-0.9cm}
\end{figure}

\noindent
% Hence,
Suy ra,
\begin{equation*}
    s_2^2 - s_1^2 = (s_2 + s_1) (s_2 - s_1) = 2xd.
\end{equation*}

% It will be established somewhat later that to obtain a distinguishable interference pattern, the distance between the sources $d$ must be considerably smaller than the distance to the screen $l$.
% The distance $x$ within whose limits interference fringes are formed is also considerably smaller than $l$.
% In these conditions, we can assume that $s_2+s_1 \approx 2l$.
% Thus, $s_2-s_1=xd/l$.
% Multiplying $s_2-s_1$ by the refractive index of the medium $n$, we get the difference in the optical path
Sau này sẽ xác định được rằng để có được một mẫu giao thoa có thể phân biệt được, khoảng cách giữa các nguôn $d$ phải nhỏ hơn đáng kể so với khoảng cách đến màn $l$.
Khoảng cách $x$ mà trong đso các vân giao thoa được hình thành cũng phải nhỏ hơn đáng kể so với $l$.
Trong điều kiện này, ta có thể cho rằng $s_2+s_1 \approx 2l$.
Do đó, $s_2-s_1=xd/l$.
Lấy tích của $s_2-s_1$ với chiết suất môi trường $n$, ta được độ lệch quang trình.
\begin{equation}\label{eq:17_7}
    \Delta = n \frac{xd}{l}.
\end{equation}

\noindent
% The introduction of this value of $\Delta$ into condition \eqref{eq:17_5} shows that intensity maxima will be observed at values of $x$ equal to
Việc thêm $\Delta$ vào điều kiện \eqref{eq:17_5} cho thấy rằng cường độ cực đại có thể được quan sát với $x$ bằng 
\begin{equation}\label{eq:17_8}
    \ab{x}{max} = \pm m \frac{l}{d} \lambda \quad (m = 0, 1, 2, \ldots).
\end{equation}

\noindent
% Here $\lambda=\lambda_0/n$ is the wavelength in the medium filling the space between the sources and the screen.
Tại đây, $\lambda=\lambda_0/n$ là bước sóng trong môi trường giữa nguồn và màn.

% Using the value of $\Delta$ given by \eqn{17_7} in condition \eqref{eq:17_6}, we get the coordinates of the intensity minima:
Sử dụng giá trị $\Delta$ cho bởi \eqn{17_7} trong điều kiện \eqref{eq:17_6}, ưe được tọa độ của cờng độ cực tiểu:
\begin{equation}\label{eq:17_9}
    \ab{x}{min} = \pm \parenthesis{m + \frac{1}{2}} \frac{l}{d} \lambda \quad (m = 0, 1, 2, \ldots).
\end{equation}

% Let us call the distance between two adjacent intensity maxima the \textbf{distance between interference fringes}, and the distance between
% adjacent intensity minima the \textbf{width of an interference fringe}.
% It can be seen from \eqns{17_8}{17_9} that the distance between fringes and the width of a fringe have the same value equal to
Gọi khoảng cách giữa 2 cường độ cực đại kề nhau là \textbf{khoảng cách giữa hai vân giao thoa}, và khoảng cách giữa hai cường độ cực tiểu kề nhau là \textbf{chiều rộng của vân giao thoa}.
Có thể thấy từ \eqns{17_8}{17_9} rằng khoảng cách giữa các vân và chiều rộng của vân có cùng giá trị và bằng 
\begin{equation}\label{eq:17_10}
    \Delta{x} = \frac{l}{d} \lambda.
\end{equation}

% According to \eqn{17_10}, the distance between the fringes grows with a decreasing distance $d$ between the sources.
% If $d$ were comparable with $l$, the distance between the fringes would be of the same order as $\lambda$, \ie, would be several scores of micrometres.
% In this case, the separate fringes would be absolutely indistinguishable.
% For an interference pattern to become distinct, the above-mentioned condition $d\ll l$ must be observed.
Theo như \eqn{17_10}, khoảng cách giữa các vân tăng lên khi khoảng cách giữa các nguồn $d$ giảm.
Nếu so sánh $d$ và $l$ thì khoảng cách giữa các vân sẽ có cùng bậc với $\lambda$, \ie, sẽ là vài micrômét.
Trong trường hợp này, các vân riêng biệt sẽ không thể nhận dạng được.
Để có thể nhìn rõ các vân giao thoa, điều kiện trên $d\ll l$ phải được thỏa mãn.

% If the intensity of the interfering waves is the same ($I_1=I_2=I_0$), then according to \eqn{17_2} the resultant intensity at the points
% for which the phase difference is $\delta$ is determined by the expression
Nếu cường độ của các sóng giao thoa là bằng nhau ($I_1=I_2=I_0$), thì, theo như \eqn{17_2}, cường độ tổng hợp tại những điểm có độ lệch pha $\delta$ được xác định bằng công thức
\begin{equation*}
    I = 2 I_0 (1 + \cos\delta) = 4 I_0 \cos^2\parenthesis{\frac{\delta}{2}}.
\end{equation*}

\noindent
% Since $\delta$ is proportional to $\Delta$ [see \eqn{17_3}], then, in accordance with \eqn{17_7}, $\delta$ grows proportionally to $x$.
% Hence, the intensity varies along the screen in accordance with the law of cosine square.
% The right-hand part of \fig{17_2} shows the dependence of $I$ on $x$ obtained in monochromatic light.
Bởi vì $\delta$ tỉ lệ thuận với $\Delta$ [xem \eqn{17_3}], nên, theo đúng với \eqn{17_7}, $\delta$ tỉ lệ thuận với $x$.
Do đó, cường độ thay đổi dọc theo màn theo định lý cosin bình phương.
Phần bên phải của \fig{17} cho thấy sự phục thuộc của $I$ vào $x$ thu được trong ánh sáng đơn sắc.

% The width of the interference fringes and their spacing depend on the wavelength $\lambda$.
% The maxima of all wavelengths will coincide only at the centre of a pattern when $x=0$.
% With an increasing distance from the centre of the pattern, the maxima of different colours become displaced from one another more and more.
% The result is blurring of the interference pattern when it is observed in white light.
% The number of distinguishable interference fringes appreciably grows in monochromatic light.
Chiều rộng của các vân giao thoa và khoảng cách giữa các vân phụ thuộc vào bước sóng $\lambda$.
Cực đại của các bước sóng sẽ trùng tại tâm của mẫu khi $x=0$.
Với khoảng cách tăng dần từ trung tâm, cực đại của các màu khác nhau sẽ ngày càng cách xa nhau hơn.
Hệ quả của việc này là hiện tượng nhòe mẫu giao thoa thu được với ánh sáng trắng.
Số lượng các vân giao thoa phân biệt được tăng đáng kể trong ánh sáng đơn sắc.

% Having measured the distance between the fringes $\Delta{x}$ and knowing $l$ and $d$, we can use \eqn{17_10} to find $\lambda$.
% It is exactly from experiments involving the interference of light that the wavelengths for light rays of various colours were determined for the first time.
Khi đã đo được khoảng cách giữa các vân $\Delta{x}$ và biết $l$ và $d$, ta có thể dùng \eqn{17_10} để tìm $\lambda$.
Đấy chính xác là những gì từ thí nghiệm giao thoa ánh sáng mà bước sóng của các tia sáng có nhiều màu sắc khác nhau đã được xác định lần đầu tiên.

% We have considered the interference of two cylindrical waves.
% Let us see what happens when two plane waves are superposed.
% Assume that the amplitudes of these waves are the same, and the directions of their propagation make the angle $2\varphi$ (\fig{17_3}).
% We shall consider that the directions of oscillations of the light vector are perpendicular to the plane of the drawing.
% The wave vectors $\vec{k}_1$ and $\vec{k}_2$ are in the plane of the drawing and have the same magnitude equal to $k=2\pi/\lambda$.
% Let us write the equations of these waves:
Ta đã xác định được sự giao thoa của hai sóng hình trụ.
Bây giờ, ta hãy xem điều gì xảy ra khi hai sóng phẳng chồng chập lên nhau.
Giả sử rằng biên độ của các sóng này là như nhau và hướng truyền của chúng tạo thành góc $2\varphi$ (\fig{17_3}).
Ta sẽ xem xét rằng hướng dao động của vector ánh sáng vuông góc với mặt phẳng của hình vẽ.
Các vectơ sóng $\vec{k}_1$ và $\vec{k}_2$ nằm trong mặt phẳng của hình vẽ và có cùng độ lớn bằng $k=2\pi/\lambda$.
Ta sẽ viết các phương trình của các sóng này:
\begin{align*}
    A \cos(\omega t - \vec{k}_1\ccdot\vec{r}) & = A \cos(\omega t - k \sin\varphi \times x - k \cos\varphi \times y),\\
    A \cos(\omega t - \vec{k}_2\ccdot\vec{r}) & = A \cos(\omega t + k \sin\varphi \times x - k \cos\varphi \times y).
\end{align*}

\begin{figure}[!htb]
	\begin{center}
		\includegraphics[scale=1]{figures/ch_17/fig_17_3.pdf}
		\caption[]{}
        % \caption[]{Superposition of two plane waves with the same amplitudes, and the directions of their propagation making the angle $2\varphi$.}
		\label{fig:17_3}
	\end{center}
	\vspace{-0.9cm}
\end{figure}

% The resultant oscillation at points with the coordinates $x$ and $y$ has the form
Dao động tổng hợp tại các điểm có tọa độ $x$ và $y$ có dạng
\begin{align}
    A \cos(\omega t - k \sin\varphi \times x &- k \cos\varphi \times y) + A \cos(\omega t + k \sin\varphi \times x - k \cos\varphi \times y)\nonumber\\
    & = 2A \cos(k \sin\varphi \times x) \cos(\omega t - k \cos\varphi \times y). \label{eq:17_11}
\end{align}

% It follows from this equation that at points where $k \sin\varphi \times x = \pm m\pi (m = 0, 1, 2, \ldots)$, the amplitude of the oscillations
% is $2A$; where $k \sin\varphi \times x = \pm (m+1/2)\pi$, the amplitude of the oscillations
% is zero.
% No matter where we place screen Sc, which is perpendicular to the $y$-axis, we shall observe on it a system of alternating light and dark fringes parallel to the $z$-axis (this axis is perpendicular to the plane of the drawing).
% The coordinates of the intensity maxima will be
Từ phương trình này suy ra rằng tại các điểm mà $k \sin\varphi \times x = \pm m\pi (m = 0, 1, 2, \ldots)$, biên độ dao động
là $2A$; tại đó $k \sin\varphi \times x = \pm (m+1/2)\pi$, biên độ dao động bằng không.
Bất kể ta đặt màn Sc ở đâu, vuông góc với trục $y$, ta sẽ quan sát thấy trên đó một dãy các vân sáng và vân tối xen kẽ song song với trục $z$ (trục này vuông góc với mặt phẳng của hình vẽ).
Tọa độ của cường độ cực đại sẽ là
\begin{equation}\label{eq:17_12}
    \ab{x}{max} = \pm \frac{m \pi}{k \sin\varphi} = \pm \frac{m \lambda}{2 \sin\varphi}.
\end{equation}

% Only the phase of the oscillations depends on the position of the screen (on the coordinate $y$) [see \eqn{17_11}].
Chỉ có pha dao động phụ thuộc vào vị trí của màn (trên tọa độ $y$) [xem \eqn{17_11}].

% We have assumed for simplicity that the initial phases of interfering waves are zero.
% If the difference between these phases is other than zero, a constant addend will appear in \eqn{17_12}---the fringe pattern will move along the screen.
Để đơn giản hóa, ta giả định rằng các pha ban đầu của sóng giao thoa là bằng không.
Nếu độ lệch pha khác không, một hằng số sẽ xuất hiện trong \eqn{17_12}---mẫu vân sẽ di chuyển dọc theo màn.

% \section{Coherence}\label{sec:17_2}
\section{Tính kết hợp}\label{sec:17_2}

% By \textbf{coherence} is meant the coordinated proceeding of several oscillatory or wave processes.
% The degree of coordination may vary.
% We can accordingly introduce the concept of the \textbf{degree of coherence} of two waves.
\textbf{Tính kết hợp} có nghĩa là tiến trình phối hợp của một số quá trình dao động hoặc sóng.
Mức độ phối hợp có thể thay đổi.
Theo đó, chúng ta có thể giới thiệu khái niệm \textbf{mức độ kết hợp} của hai sóng.

% \textbf{Temporal} and \textbf{spatial coherence} are distinguished.
% We shall begin with a discussion of temporal coherence.
Cần phân biệt giữa \textbf{Nhất quán thời gian} (Temporal coherence) và \textbf{Nhất quán không gian} (Spacial coherence).
Ta bắt đầu nói về nhất quán thời gian thời gian.

% \textbf{Temporal Coherence.}
% The process of interference described in the preceding section is idealized.
% This process is actually much more complicated.
% The reason is that a monochromatic wave described
% by the expression
\textbf{Nhất quán thời gian.}
Quá trình giao thoa được mô tả trong phần trước đã được lý tưởng hóa.
Quá trình này thực sự phức tạp hơn nhiều.
Lý do là sóng đơn sắc được mô tả bởi biểu thức
\begin{equation*}
    A \cos(\omega t - kr + \alpha),
\end{equation*}

\noindent
% where $A$, $\omega$, and $\alpha$ are constants, is an abstraction.
% A real light wave is formed by the superposition of oscillations of all possible frequencies (or wavelengths) confined within a more or less narrow but finite range of frequencies $\Delta{\omega}$ (or the corresponding range of wavelengths $\Delta{\lambda}$).
% Even for light considered to be monochromatic (single-coloured), the frequency interval $\Delta{\omega}$ is finite\footnote{The spectral lines emitted by atoms have a ``natural'' width of the order of \SI{e-8}{\radian\per\second} ($\Delta{\lambda}\sim\SI{e-4}{\angstrom}$).}.
% In addition, the amplitude of the wave $A$ and the phase $\alpha$ undergo continuous random (chaotic) changes with time.
% Hence, the oscillations produced at a certain point of space by two superposed light waves have the form
trong đó $A$, $\omega$ và $\alpha$ là hằng số.
Sóng ánh sáng thực được hình thành bằng các chồng chập của dao động trong các tần số khả thi (hoặc bước sóng) và bị giới hạn trong một phạm vi tần số $\Delta{\omega}$ hữu hạn (hoặc phạm vi bước sóng tương ứng $\Delta{\lambda}$).
Đối với ánh sáng được coi là đơn sắc (một màu), khoảng tần số $\Delta{\omega}$ là hữu hạn\footnote{Các vạch quang phổ do nguyên tử phát ra có độ rộng ``tự nhiên'' khoảng \SI{e-8}{\radian\per\second} ($\Delta{\lambda}\sim\SI{e-4}{\angstrom}$).}.
Ngoài ra, biên độ của sóng $A$ và pha $\alpha$ biến thiên ngẫu nhiên (hỗn loạn) liên tục theo thời gian.
Do đó, các dao động tại một điểm trong không gian được tạo ra bởi hai sóng ánh sáng chồng lên nhau có dạng
\begin{equation}\label{eq:17_13}
    A_1(t) \cos[\omega_1(t) t + \alpha_1(t)],\quad A_2(t) \cos[\omega_2(t) t + \alpha_2(t)],
\end{equation}

\noindent
% the chaotic changes in the functions $A_1(t)$, $\omega_1(t)$, $\alpha_1(t)$, $A_2(t)$, $\omega_2(t)$, and $\alpha_2(t)$ being absolutely independent.
sự biến đổi hỗn loạn trong các hàm $A_1(t)$, $\omega_1(t)$, $\alpha_1(t)$, $A_2(t)$, $\omega_2(t)$, và $\alpha_2(t)$ hoàn toàn độc lập.

% We shall assume for simplicity's sake that the amplitudes $A_1$ and $A_2$ are constant.
% Changes in the frequency and phase can be reduced either to a change only in the phase, or to a change only in the frequency.
% Let us write the function
Để đơn giản hóa vấn đề, chúng ta sẽ giả định rằng biên độ $A_1$ và $A_2$ là hằng số.
Những thay đổi về tần số và pha có thể được giảm thành thay đổi về pha hoặc thay đổi về tần số.
Từ đó ta có hàm
\begin{equation}\label{eq:17_14}
    f(t) = A \cos[\omega(t) t + \alpha(t)],
\end{equation}

\noindent
% in the form
dưới dạng
\begin{equation*}
    f(t) = A \cos\{\omega_0 t + [\omega(t) - \omega_0] t + \alpha(t)\},
\end{equation*}

\noindent
% where $\omega_0$ is a certain average value of the frequency, and introduce the notation $[\omega(t) - \omega_0] t + \alpha(t) = \alpha'(t)$.
% Equation \eqref{eq:17_14} will, thus, become
trong đó $\omega_0$ là giá trị trung bình của tần số và ta có $[\omega(t) - \omega_0] t + \alpha(t) = \alpha'(t)$.
Do đó, phương trình \eqref{eq:17_14} sẽ trở thành
\begin{equation}\label{eq:17_15}
    f(t) = A \cos[\omega_0 t + \alpha'(t)].
\end{equation}

\noindent
% We have obtained a function in which only the phase of the oscillation changes chaotically.
Như vậy, ta đã có được hàm số mà chỉ có pha dao động biến thiên hỗn loạn.

% On the other hand, it is proved in mathematics that an inharmonic function, for example, function \eqref{eq:17_14}, can be represented in the form of the sum of harmonic functions with frequencies confined within a certain interval $\Delta(\omega)$ [see \eqn{17_16}].
Mặt khác, toán học đã chứng minh rằng một hàm không điều hòa, ví dụ như hàm \eqref{eq:17_14}, có thể được biểu diễn dưới dạng tổng các hàm điều hòa có tần số giới hạn trong một khoảng nhất định $\Delta(\omega)$ [xem \eqn{17_16}].

% Thus, when considering the matter of coherence, two approaches are possible: a ``phase'' one and a ``frequency'' one.
% Let us begin with the phase approach.
% Assume that the frequencies $\omega_1$ and $\omega_2$ in Eqs. \eqref{eq:17_13} satisfy the condition $\omega_1 = \omega_2 = \text{constant}$.
% Now let us find the influence of a change in the phases $\alpha_1$ and $\alpha_2$.
% According to \eqn{17_12}, with our assumptions, the intensity of light at a given point is determined by the expression
Do đó, khi xem xét vấn đề về tính kết hợp, ta có thể có hai cách tiếp cận: về ``pha'' và về ``tần số''.
Chúng ta hãy bắt đầu với cách tiếp cận pha.
Giả sử rằng tần số $\omega_1$ và $\omega_2$ trong các phương trình \eqref{eq:17_13} thỏa mãn điều kiện $\omega_1 = \omega_2 = \text{hằng số}$.
Ta sẽ tìm ảnh hưởng của sự thay đổi trong các pha $\alpha_1$ và $\alpha_2$.
Theo \eqn{17_12}, với các giả định, cường độ ánh sáng tại một điểm nhất định được xác định bởi biểu thức
\begin{equation*}
    I = I_1 + I_2 + 2 \sqrt{I_1 I_2} \cos[\delta(t)],
\end{equation*}

\noindent
% where $\delta(t) = \alpha_2(t) - \alpha_1(t)$.
% The last addend in this equation is called the \textbf{interference term}.
Với $\delta(t) = \alpha_2(t) - \alpha_1(t)$.
Số hạng cuối của phương trình này được gọi là \textbf{số hạng giao thoa}

% An instrument that can be used to observe an interference pattern (the eye\footnote{We remind our reader that the showing of motion picture films is based on the inertia of visual perception, which is about \SI{0.1}{\second}.}, a photographic plate, etc.) has a certain inertia.
% In this connection, it registers a pattern averaged over the time interval $\ab{t}{instr}$ needed for ``operation'' of the instrument.
% If during the time $\ab{t}{instr}$ the factor $\cos[\delta(t)]$ takes on all the values from $-1$ to $+1$, the average value of the interference term will be zero.
% Therefore, the intensity registered by the instrument will equal the sum of the intensities produced at a given point by each of the waves separately---interference is absent, and we are forced to acknowledge that the waves are incoherent.
Một công cụ có thể được sử dụng để quan sát một mẫu giao thoa (mắt\footnote{Tôi xin nhắc lại rằng việc chiếu phim ảnh phụ thuộc vào thời gian nhận thức của thị giác, khoảng \SI{0.1}{\second}.}, một tấm ảnh, v.v.) có độ trì hoãn nhất định.
Về mặt này, dụng cụ ghi lại một mẫu trung bình trong khoảng thời gian $\ab{t}{instr}$ cần thiết cho ``hoạt động'' của công cụ.
Nếu trong thời gian $\ab{t}{instr}$, hệ số $\cos[\delta(t)]$ có tất cả các giá trị từ $-1$ đến $+1$, thì giá trị trung bình của số hạng giao thoa sẽ bằng không.
Do đó, cường độ mà công cụ ghi lại sẽ bằng tổng cường độ tạo ra tại một điểm nhất định bởi từng sóng riêng biệt --- không có giao thoa và chúng ta buộc phải thừa nhận rằng các sóng không kết hợp.

% If during the time $\ab{t}{instr}$, however, the value of $\cos[\delta(t)]$ remains virtually constant\footnote{The phase difference $\delta(t)$ varies for different points of space. The influence of the interference term manifests itself at the points where it differs from zero.}, the instrument will detect interference, and the waves must be acknowledged as coherent.
Tuy nhiên, trong thời gian $\ab{t}{instr}$, giá trị của $\cos[\delta(t)]$ vẫn hầu như không đổi\footnote{Độ lệch pha $\delta(t)$ thay đổi đối với các điểm khác nhau trong không gian. Ảnh hưởng của số hạng giao thoa thể hiện tại các điểm mà nó khác không.}, thiết bị sẽ phát hiện giao thoa và các sóng phải được xác nhận là đã kết hợp.

% It follows from the above that the concept of coherence is relative: two waves can behave like coherent ones when observed using one instrument (having a low inertia), and like incoherent ones when observed using another instrument (having a high inertia).
% The coherent properties of waves are characterized by introducing the \textbf{coherence time} $\ab{t}{coh}$.
% It is defined as the time during which a chance change in the wave phase $\alpha(t)$ reaches a value of the order of $\pi$.
% During the time $\ab{t}{coh}$, an oscillation, as it were, forgets its initial phase and becomes incoherent with respect to itself.
Từ những điều trên, ta suy ra rằng khái niệm về tính kết hợp là tương đối: hai sóng có thể hoạt động giống như những sóng kết hợp khi quan sát bằng một dụng cụ (có độ trì hoãn thấp) và giống như những sóng không kết hợp khi quan sát bằng một dụng cụ khác (có độ trì hoãn cao).
Các tính chất kết hợp của sóng được đặc trưng bằng cách áp dụng \textbf{thời gian kết hợp} $\ab{t}{coh}$.
Nó được định nghĩa là thời gian mà sự thay đổi ngẫu nhiên trong pha sóng $\alpha(t)$ đạt đến giá trị bậc $\pi$.
Trong thời gian $\ab{t}{coh}$, một dao động, như thể, bỏ qua pha ban đầu của nó và trở nên không kết hợp đối với chính nó.

% Using the concept of the coherence time, we can say that when the instrument time is much greater than the coherence time of the superposed waves ($\ab{t}{instr}\gg\ab{t}{coh}$), the instrument does not register interference.
% When $\ab{t}{instr}\ll\ab{t}{coh}$, the instrument will detect a sharp interference pattern.
% At intermediate values of $\ab{t}{instr}$, the sharpness of the pattern will diminish as $\ab{t}{instr}$ grows from values smaller than
% $\ab{t}{coh}$ to values greater than it.
Sử dụng khái niệm thời gian kết hợp, chúng ta có thể nói rằng khi thời gian của thiết bị lớn hơn nhiều so với thời gian kết hợp của các sóng chồng chất ($\ab{t}{instr}\gg\ab{t}{coh}$), thiết bị không ghi nhận được sự giao thoa.
Khi $\ab{t}{instr}\ll\ab{t}{coh}$, thiết bị sẽ phát hiện ra một mẫu giao thoa sắc nét.
Ở các giá trị trung gian của $\ab{t}{instr}$, độ sắc nét của mẫu sẽ giảm dần khi $\ab{t}{instr}$ tăng từ các giá trị nhỏ hơn
$\ab{t}{coh}$ lên các giá trị lớn hơn.

% The distance $\ab{l}{coh} = c\ab{t}{coh}$ over which a wave travels during the time $\ab{l}{coh}$ is called the \textbf{coherence length} (or the \textbf{train length}).
% The coherence length is the distance over which a chance change in the phase reaches a value of about $\pi$.
% To obtain an interference pattern by splitting a natural wave into two parts, it is essential that the optical path difference $\Delta$ be smaller than the coherence length.
% This requirement limits the number of visible interference fringes observed when using the layout shown in \fig{17_2}.
% An increase in the fringe number $m$ is attended by a growth in the path difference.
% As a result, the sharpness of the fringes becomes poorer and poorer.
Khoảng cách $\ab{l}{coh} = c\ab{t}{coh}$ mà sóng truyền đi trong thời gian $\ab{l}{coh}$ được gọi là \textbf{độ dài kết hợp} (hoặc \textbf{độ dài chuỗi}).
Độ dài kết hợp là khoảng cách mà sự thay đổi ngẫu nhiên trong pha đạt đến giá trị khoảng $\pi$.
Để có được mẫu giao thoa bằng cách chia sóng tự nhiên thành hai phần, độ lệch đường quang $\Delta$ phải nhỏ hơn độ dài kết hợp.
Yêu cầu này giới hạn số lượng vân giao thoa có thể nhìn thấy được khi sử dụng bố cục được hiển thị trong \fig{17_2}.
Sự gia tăng số vân $m$ đi kèm với sự gia tăng độ lệch đường đi.
Kết quả là, độ sắc nét của các vân ngày càng kém hơn.

% Let us pass over to a consideration of the part of the non-monochromatic nature of light waves.
% Assume that light consists of a sequence of identical trains of frequency $\omega_0$ and duration $T$.
% When one train is replaced with another one, the phase experiences disordered changes.
% As a result, the trains are mutually incoherent.
% With these assumptions, the duration of a train $\tau$ virtually coincides with the coherence
% time $\ab{t}{coh}$.
Chúng ta hãy chuyển sang xem xét bản chất phi đơn sắc của sóng ánh sáng.
Giả sử rằng ánh sáng bao gồm một chuỗi giống hệt nhau có tần số $\omega_0$ và thời lượng $T$.
Khi một chuỗi được thay thế bằng một chuỗi khác, pha sẽ thay đổi hỗn loạn.
Kết quả là, các chuỗi không nhất quán với nhau.
Với những giả định này, thời lượng của một chuỗi $\tau$ thực tế trùng với thời gian kết hợp $\ab{t}{coh}$.

% In mathematics, the Fourier theorem is proved, according to which any finite and integrable function $F(t)$ can be represented in the form of the sum of an infinite number of harmonic components with a continuously changing frequency:
Trong toán học, định lý Fourier đã được chứng minh, theo đó bất kỳ hàm hữu hạn và hàm tích phân $F(t)$ nào cũng có thể được biểu diễn dưới dạng tổng của vô số thành phần điều hòa có tần số thay đổi liên tục:
\begin{equation}\label{eq:17_16}
    F(t) = \int_{-\infty}{+\infty} A(\omega) \, e^{i\omega t}\, \deriv{\omega}.
\end{equation}

\noindent
% Expression \eqref{eq:17_16} is known as the \textbf{Fourier integral}.
% The function $A(\omega)$ inside the integral is the amplitude of the relevant monochromatic component.
% According to the theory of Fourier integrals, the analytical form of the function $A(\omega)$ is determined by the expression
Biểu thức \eqref{eq:17_16} được gọi là \textbf{Tích phân Fourier}.
Hàm $A(\omega)$ bên trong tích phân là biên độ của thành phần đơn sắc.
Theo nguyên lý tích phân Fourier, hàm $A(\omega)$ được xác định bởi biểu thức
\begin{equation}\label{eq:17_17}
    A(\omega) = 2\pi \int_{-\infty}^{+\infty} F(\xi)\, e^{-i\omega \xi}\, \deriv{\xi},
\end{equation}

\noindent
% where $\xi$ is an auxiliary integration variable.

% Assume that the function $F(t)$ describes a light disturbance at a certain point at the moment of time $t$ due to a single wave train.
% Hence, it is determined by the conditions
trong đó $\xi$ là biến tích phân phụ trợ.

Giả sử hàm $F(t)$ mô tả nhiễu động ánh sáng tại một điểm nhất định tại thời điểm $t$ do một chuỗi sóng đơn lẻ gây ra.
Do đó, nó được xác định bởi các điều kiện
\begin{align*}
    F(t) &= A_0\, \exp(i \omega_0 t) \quad \text{at } |t| \ll \frac{\tau}{2} \\
    F(t) &= 0 \quad\quad\quad\quad\quad\quad\! \text{at } |t| > \frac{\tau}{2}.
\end{align*}

\noindent
% A graph of the real part of this function is given in \fig{17_4}.
Có thể nhìn thấy một phần biểu đồ của hàm số này ở \fig{17_4}.

\begin{figure}[!htb]
	\begin{center}
		\includegraphics[scale=1]{figures/ch_17/fig_17_4.pdf}
		\caption[]{}
        % \caption[]{Real part of function $F(t)$.}
		\label{fig:17_4}
	\end{center}
	\vspace{-0.9cm}
\end{figure}

% Outside the interval from $-\tau/2$ to $+\tau/2$, the function $F(t)$ is zero.
% Therefore, expression \eqref{eq:17_17} determining the amplitude of the harmonic components has the form
Bên ngoài khoảng từ $-\tau/2$ đến $+\tau/2$, hàm $F(t)$ bằng không.
Do đó, biểu thức \eqref{eq:17_17} xác định biên độ của các thành phần điều hòa có dạng
\begin{align*}
    A(\omega) &= 2\pi \int_{-\tau/2}^{+\tau/2} [A_0 \exp(i\omega_0\xi)] \exp(-i\omega\xi)\, \deriv{\xi}\\
    &= 2\pi A_0 \int_{-\tau/2}^{+\tau/2} \exp[i(\omega_0-\omega)\xi]\, \deriv{\xi} = 2\pi A_0 \left.\frac{\exp[i(\omega_0-\omega)\xi]}{i(\omega_0-\omega)}\right|^{+\tau/2}_{-\tau/2}.
\end{align*}

\noindent
% After introducing the integration limits and simple transformations, we arrive at the equation
Sau khi giới thiệu các giới hạn tích phân và các phép biến đổi đơn giản, ta có phương trình
\begin{equation*}
    A(\omega) = \pi A_0 \tau\, \frac{\sin[(\omega-\omega_0)\tau/2]}{(\omega-\omega_0)\tau/2}.
\end{equation*}

% The intensity $I(\omega)$ of a harmonic wave component is proportional to the square of the amplitude, \ie, to the expression
Cường độ $I(\omega)$ của thành phần sóng điều hòa tỉ lệ thuận với bình phương biên độ, \ie, với biểu thức
\begin{equation}\label{eq:17_18}
    f(\omega) = \frac{\sin^2[(\omega-\omega_0)\tau/2]}{[(\omega-\omega_0)\tau/2]^2}.
\end{equation}

\noindent
% A graph of function \eqref{eq:17_18} is shown in \fig{17_5}.
% A glance at the figure shows that the intensity of the components whose frequencies are within the interval of width $\Delta{\omega} = 2\pi/\tau$ considerably exceeds the intensity of the remaining components.
% This circumstance allows us to relate the duration of a train $\tau$ to the effective frequency range $\Delta{\omega}$ of a Fourier spectrum:
Đồ thị của hàm \eqref{eq:17_18} được hiển thị trong \fig{17_5}.
Nhìn hình ta thấy cường độ của các thành phần có tần số nằm trong khoảng $\Delta{\omega} = 2\pi/\tau$ lớn hơn đáng kể cường độ của các thành phần còn lại.
Việc này cho phép ta liên hệ thời gian của một chuỗi $\tau$ với dải tần số hiệu dụng $\Delta{\omega}$ của phổ Fourier:
\begin{equation*}
    \tau = \frac{2\pi}{\Delta{\omega}} = \frac{1}{\Delta{\nu}}.
\end{equation*}

\begin{figure}[!htb]
	\begin{center}
		\includegraphics[scale=1]{figures/ch_17/fig_17_5.pdf}
		\caption[]{}
        % \caption[]{Graph depicting function $f(\omega)$ \eqref{eq:17_18}.}
		\label{fig:17_5}
	\end{center}
	\vspace{-0.9cm}
\end{figure}

% Identifying $\tau$ with the coherence time, we arrive at the relation
Liên hệ $\tau$ với thời gian kết hợp, ta có tỉ lệ
\begin{equation}\label{eq:17_19}
    \ab{t}{coh} \sim \frac{1}{\Delta{\nu}}
\end{equation}

\noindent
% (The sign $\sim$ stands for ``equal to in the order of magnitude'').
(Dấu $\sim$ có nghĩa là ``bằng theo tỉ lệ độ lớn'').

% It can be seen from expression \eqref{eq:17_19} that the broader the interval of frequencies present in a given light wave, the smaller is the coherence time of this wave.
Có thể thấy từ biểu thức \eqref{eq:17_19} rằng hiệu tần số của sóng ánh sáng càng lớn thì thời gian kết hợp của sóng này càng nhỏ.

% The frequency is related to the wavelength in a vacuum by the expression $\nu=c/\lambda_0$.
% Differentiation of this expression yields $\Delta{\nu}=c\Delta{\lambda_0}/\lambda_0^2 \approx c \Delta{\lambda}/\lambda^2$ (we have omitted the minus sign obtained in differentiation and also assumed that $\lambda_0\approx\lambda$).
% Substituting for $\Delta{\nu}$ in \eqn{17_19} its expression through $\lambda$ and $\Delta{\lambda}$, we obtain the following expression for the coherence time:
Tần số có liên hệ đến bước sóng trong chân không theo biểu thức $\nu=c/\lambda_0$.
Vi phân của biểu thức này cho kết quả $\Delta{\nu}=c\Delta{\lambda_0}/\lambda_0^2 \approx c \Delta{\lambda}/\lambda^2$ (ta đã bỏ dấu trừ trong phép vi phân và cũng giả định rằng $\lambda_0\approx\lambda$).
Thay $\Delta{\nu}$ trong \eqn{17_19} bằng $\lambda$ và $\Delta{\lambda}$, ta thu được biểu thức sau cho thời gian kết hợp:
\begin{equation}\label{eq:17_20}
    \ab{t}{coh} \sim \frac{\lambda^2}{c \Delta{\lambda}}.
\end{equation}

\noindent
% Hence, we get the following value for the coherence length:
Từ đó, ta rút ra được giá trị của độ dài kết hợp.
\begin{equation}\label{eq:17_21}
    \ab{l}{coh} = c \ab{t}{coh} \sim \frac{\lambda^2}{\Delta{\lambda}}.
\end{equation}

% Examination of \eqn{17_5} shows that the path difference at which a maximum of the $m$-th order is obtained is determined by the relation
Xem xét \eqn{17_5}, ta nhận thấy được độ lệch quang trình mà tại đó giá trị thứ $m$ đạt cực đại được xác định bởi biểu thức
\begin{equation*}
    \Delta_m = \pm m \lambda_0 \approx \pm m \lambda.
\end{equation*}

\noindent
% When this path difference reaches values of the order of the coherence length, the fringes become indistinguishable.
% Consequently, the extreme interference order observed is determined by the condition
Khi độ lệch này đạt đến các giá trị theo thứ tự của độ dài kết hợp, các vân trở nên không thể phân biệt được.
Do đó, thứ tự giao thoa cực đại được quan sát được xác định bởi điều kiện
\begin{equation*}
    \ab{m}{extr} \lambda \sim \ab{l}{coh} \sim \frac{\lambda^2}{\Delta{\lambda}},
\end{equation*}

\noindent
% whence
với
\begin{equation}\label{eq:17_22}
    \ab{m}{extr} \sim \frac{\lambda}{\Delta{\lambda}}.
\end{equation}

\noindent
% It follows from \eqn{17_22} that the number of interference fringes observed according to the layout shown in \fig{17_2} grows when the wavelength interval in the light used diminishes.
Từ \eqn{17_22} suy ra rằng số vân giao thoa quan sát được theo sơ đồ trong \fig{17_2} tăng lên khi khoảng cách bước sóng ánh sáng giảm đi.

\textbf{Nhất quán không gian}
% According to the equation $k=\omega/v = n\omega/c$, scattering of the frequencies $\Delta{\omega}$ results in scattering of the values of $k$.
% We have established that the temporal coherence is determined by the value of $\Delta{\omega}$.
% Consequently, the temporal coherence is associated
% with scattering of the values of the magnitude of the wave vector $\vec{k}$.
% Spatial coherence is associated with scattering of the directions of the vector $\vec{k}$ that is characterized by the quantity $\Delta{\vecuni{k}}$.
Theo phương trình $k=\omega/v = n\omega/c$, sự phân bố của các tần số $\Delta{\omega}$ dẫn đến sự phân bố của các giá trị $k$.
Ta biết rằng sự nhất quán thời gian được xác định bởi giá trị của $\Delta{\omega}$.
Do đó, sự nhất quán thời gian có liên quan với sự phân bố của các giá trị độ lớn của vectơ sóng $\vec{k}$.
Sự nhất quán không gian có liên quan đến sự phân bố của hướng vectơ $\vec{k}$ được đặc trưng bởi đại lượng $\Delta{\vecuni{k}}$.

% The setting up at a certain point of space of oscillations produced by waves with different values of $\vecuni{k}$ is possible if these waves are emitted by different sections of an extended (not a point) light source.
% Let us assume for simplicity's sake that the source has the form of a disk visible from a given point at the angle $\varphi$.
% It can be seen from \fig{17_6} that the angle $\varphi$ characterizes the interval confining the unit vectors $\vecuni{k}$.
% We shall consider that this angle is small.
Việc thiết lập các dao động do sóng có giá trị $\vecuni{k}$ khác nhau tại một điểm nhất định là có thể nếu các sóng này được phát ra từ các phần khác nhau của một nguồn sáng không phải là chất điểm.
Ta hãy giả sử rằng nguồn sáng ấy có dạng đĩa và có thể được nhìn thấy từ một điểm nhất định ở góc $\varphi$.
Có thể thấy từ \fig{17_6} rằng góc $\varphi$ đặc trưng cho khoảng giới hạn của các vectơ đơn vị $\vecuni{k}$.
Chúng ta sẽ coi góc này là nhỏ.

\begin{figure}[!htb]
	\begin{center}
		\includegraphics[scale=1]{figures/ch_17/fig_17_6.pdf}
		\caption[]{}
        % \caption[]{Waves emitted by different sections of an extended source with the form of a disk visible from a given point at the angle $\varphi$.}
		\label{fig:17_6}
	\end{center}
	\vspace{-0.9cm}
\end{figure}

% Assume that the light from the source falls on two narrow slits behind which there is a screen (\fig{17_7}).
% We shall consider that the interval of frequencies emitted by the source is very small.
% This is needed for the degree of temporal coherence to be sufficient for obtaining a sharp interference pattern.
% The wave arriving from the section of the surface designated in \fig{17_7} by $0$ produces a zero-order maximum M at the middle of the screen.
% The zero-order maximum M$'$ produced by the wave arriving from section $0'$ will be displaced from the middle of the screen by the distance $x'$.
% Owing to the smallness of the angle $\varphi$ and of the ratio $d/l$, we can consider that $x' = l\varphi/2$.
% The zero-order maximum M$''$ produced by the wave arriving from section $0''$ is displaced in the opposite direction from the middle of the screen over the distance $x''$ equal to $x'$.
% The zero-order maxima from the other sections of the source will be between the maxima M$'$ and M$''$.
Giả sử rằng ánh sáng từ nguồn chiếu vào hai khe hẹp tới một màn chắn phía sau. (\fig{17_7}).
Ta sẽ cho rằng khoảng biến thiên tần số do nguồn phát ra rất nhỏ.
Điều này là cần thiết để mức độ nhất quán thời gian đủ để thu được một mẫu giao thoa sắc nét.
Sóng đến từ phần bề mặt được chỉ định bởi $0$ trong \fig{17_7} tạo ra một cực đại bậc 0 M ở giữa màn chắn.
Cực đại bậc 0 M$'$ do sóng đến từ phần $0'$ tạo ra sẽ bị dịch chuyển khỏi giữa màn chắn một khoảng cách $x'$.
Do góc $\varphi$ và tỷ số $d/l$ nhỏ, ta có thể xem rằng $x' = l\varphi/2$.
Cực đại bậc 0 M$''$ do sóng đến từ $0''$ tạo ra bị dịch chuyển theo hướng ngược lại từ giữa màn chắn trên khoảng cách $x''$ bằng $x'$.
Các cực đại bậc 0 từ các phần khác của nguồn sẽ nằm giữa các cực đại M$'$ và M$''$.

% The separate sections of the light source produce waves whose phases are in no way related to one another.
% For this reason, the interference pattern appearing on the screen will be a superposition of the patterns produced by each section separately.
% If the displacement $x'$ is much smaller than the width of an interference fringe $\Delta{x}=l\lambda/d$ [see \eqn{17_10}], then, the maxima from different sections of the source will practically be superposed on one another, and the pattern will be like the one produced by a point source.
% When $x'\approx\Delta{x}$, the maxima from some sections will coincide with the minima from others, and no interference pattern will be observed.
% Thus, an interference pattern will be distinguishable provided that $x'<\Delta{x}$, \ie,
Mỗi phần riêng biệt của nguồn sáng tạo ra các sóng có pha độc lập với mỗi sóng.
Vì lý do này, mẫu giao thoa xuất hiện trên màn hình sẽ là chồng chập của các mẫu do từng phần riêng biệt tạo ra.
Nếu độ dịch chuyển $x'$ nhỏ hơn nhiều so với chiều rộng của vân giao thoa $\Delta{x}=l\lambda/d$ [xem \eqn{17_10}], thì các cực đại từ các phần khác nhau của nguồn sẽ chồng chập lên nhau và mẫu sẽ giống như mẫu do một chất điểm phát ra.
Khi $x'\approx\Delta{x}$, cực đại của một số phần sẽ trùng với cực tiểu của các phần khác và sẽ không có giao thoa.
Do đó, có thể phân biệt được mẫu giao thoa với điều kiện là $x'<\Delta{x}$, \ie,
\begin{equation}\label{eq:17_23}
    \frac{l \varphi}{2} < \frac{l \lambda}{d},
\end{equation}

\noindent
% or
hoặc
\begin{equation}\label{eq:17_24}
    \varphi < \frac{\lambda}{d}.
\end{equation}

\noindent
% We have omitted the factor $2$ when passing over from expression \eqref{eq:17_23} to \eqref{eq:17_24}.
Ta đã loại bỏ hệ số $2$ khi chuyển từ biểu thức \eqref{eq:17_23} sang biểu thức \eqref{eq:17_24}.

\begin{figure}[!htb]
	\begin{center}
		\includegraphics[scale=1]{figures/ch_17/fig_17_7.pdf}
		\caption[]{}
        % \caption[]{Interference pattern obtained from waves emitted through two slits. The source is that from \fig{17_6}.}
		\label{fig:17_7}
	\end{center}
	\vspace{-0.9cm}
\end{figure}

% Formula \eqref{eq:17_24} determines the angular dimensions of a source at which interference is observed.
% We can also use this formula to find the greatest distance between the slits at which interference from a source with the angular dimension $\varphi$ can still be observed.
% Multiplying inequality \eqref{eq:17_24} by $d/\varphi$, we arrive at the condition
Công thức \eqref{eq:17_24} xác định góc của một nguồn mà tại đó có giao thoa.
Chúng ta cũng có thể sử dụng công thức này để tìm khoảng cách lớn nhất giữa các khe hở mà giao thoa có thể được quan sát dưới góc $\varphi$.
Nhân bất đẳng thức \eqref{eq:17_24} với $d/\varphi$, chúng ta có điều kiện
\begin{equation}\label{eq:17_25}
    d < \frac{\lambda}{\varphi}.
\end{equation}

% A collection of waves with different values of $\vecuni{k}$ can be replaced with the resultant wave falling on a screen with slits.
% The absence of an interference pattern signifies that the oscillations produced by this wave at the places where the first and second slits are situated are incoherent.
% Consequently, the oscillations in the wave itself at points at a distance $d$ apart are incoherent too.
% If the source were ideally monochromatic (this means that $\Delta{\nu}=0$ and $\ab{t}{coh}=\infty$), the surface passing through the slits would be a wave one, and the oscillations at all the points of this surface would occur in the same phase.
% We have established that when $\Delta{v}\neq 0$ and the source has finite dimensions ($\varphi\neq 0$), the oscillations at points of a surface at a distance of $d>\lambda/\varphi$ are incoherent.
Một tập hợp các sóng có giá trị $\vecuni{k}$ khác nhau có thể được thay thế bằng sóng tổng hợp trên màn chắn.
Việc thiếu mẫu giao thoa cho thấy các dao động do sóng này tạo ra không kết hợp tại vị trí khe hở thứ nhất và thứ hai.
Do đó, các dao động của sóng tại các điểm cách nhau một khoảng cách $d$ cũng không kết hợp.
Nếu nguồn lý tưởng là hoàn toàn đơn sắc (điều này có nghĩa là $\Delta{\nu}=0$ và $\ab{t}{coh}=\infty$), thì mặt sóng sẽ đi qua khe hở và các dao động tại tất cả các điểm của bề mặt này sẽ có cùng một pha.
Ta đã biết rằng khi $\Delta{v}\neq 0$ và nguồn có kích thước hữu hạn ($\varphi\neq 0$), thì các dao động tại các điểm trên bề mặt ở khoảng cách $d>\lambda/\varphi$ là không kết hợp.

% We shall call a surface, which would be a wave one if the source were monochromatic, a pseudowave surface\footnote{It must be borne in mind that this term is not used in scientific publications. The author has coined it for conditional use only to make the treatment
% more illustrative.} for brevity.
% We could satisfy condition \eqref{eq:17_24} by reducing the distance $d$ between the slits, \ie, by taking closer points of the pseudowave surface.
% Consequently, oscillations produced by a wave at adequately close points of a pseudowave surface are coherent.
% Such coherence is called \textbf{spatial}.
% Thus, the phase of an oscillation changes chaotically when passing from one point of a pseudowave surface to another.
% Let us introduce the distance $\ab{\rho}{coh}$, upon displacement by which along a pseudowave
% surface a random change in the phase reaches a value of about $\pi$.
% Oscillations at two points of a pseudowave surface spaced apart at a distance less than $\ab{\rho}{coh}$ will be approximately coherent.
% The distance $\ab{\rho}{coh}$ is called the \textbf{spatial coherence length} or the \textbf{coherence radius}.
% It can be seen from expression \eqref{eq:17_25} that
Ta sẽ gọi một bề mặt, sẽ là mặt sóng nếu như nguồn đơn sắc, là bề mặt sóng ảo\footnote{Cần lưu ý rằng thuật ngữ này không được sử dụng trong các ấn phẩm khoa học. Tác giả đã đặt ra nó nhằm tăng sự minh họa.} cho ngắn gọn.
Ta có thể thỏa mãn điều kiện \eqref{eq:17_24} bằng cách giảm khoảng cách $d$ giữa các khe, \ie, bằng cách lấy các điểm gần hơn của bề mặt sóng ảo.
Do đó, các dao động do sóng tạo ra tại các điểm đủ gần của bề mặt sóng ảo có tính kết hợp.
Tính kết hợp như thế được gọi là \textbf{nhất quán không gian}.
Do đó, pha của dao động thay đổi hỗn loạn khi đi từ điểm này của bề mặt sóng ảo sang điểm khác.
Ta hãy nói về khoảng cách $\ab{\rho}{coh}$, khi giá trị dịch chuyển trên mặt sóng ảo, độ lệch pha ngẫu nhiên đạt giá trị $\pi$.
Dao động tại hai điểm của mặt sóng ảo cách nhau một khoảng nhỏ hơn $\ab{\rho}{coh}$ sẽ gần như kết hợp.
Khoảng cách $\ab{\rho}{coh}$ được gọi là \textbf{chiều dài nhất quán không gian} (spatial coherence length) hoặc \textbf{bán kính kết hợp}.
Có thể thấy từ biểu thức \eqref{eq:17_25} rằng
\begin{equation}\label{eq:17_26}
    \ab{\rho}{coh} \sim \frac{\lambda}{\varphi}.
\end{equation}

% The angular dimension of the Sun is about $0.01$ radians, and the length of its light waves is about \SI{0.5}{\micro\metre}.
% Hence, the coherence radius of the light waves arriving from the Sun has a value of the order of
Góc của Mặt trời là khoảng $0,01$ radian, và độ dài của sóng ánh sáng là khoảng \SI{0,5}{\micro\metre}.
Do đó, bán kính kết hợp của sóng ánh sáng đến từ Mặt trời có giá trị theo thứ tự
\begin{equation}\label{eq:17_27}
    \ab{\rho}{coh} = \frac{0.5}{0.01} = \SI{50}{\micro\metre} = \SI{0.05}{\milli\metre}.
\end{equation}

% The entire space occupied by a wave can be divided into parts in each of which the wave approximately retains coherence.
% The volume of such a part of space, called the \textbf{coherence volume}, in its order of
% magnitude equals the product of the temporal coherence length and the area of a circle of radius $\ab{\rho}{coh}$.
Toàn bộ không gian chứa sóng có thể được chia thành các phần mà trong mỗi phần, sóng vẫn giữ được tính kết hợp.
Thể tích của một phần không gian như vậy, được gọi là \textbf{thể tích kết hợp}, độ lớn của nó bằng tích của độ dài nhất quán thời gian và diện tích của một hình tròn bán kính $\ab{\rho}{coh}$.

% The spatial coherence of a light wave near the surface of the heated body emitting it is restricted by a value of $\ab{\rho}{coh}$ of only a few wavelengths.
% With an increasing distance from the source, the degree of spatial coherence grows.
% The radiation of a laser\footnote{Lasers will be treated in Vol. III of our course.} has an enormous temporal and spatial coherence.
% At the outlet opening of a laser, spatial coherence is observed throughout the entire cross section of the light beam.
Sự nhất quán không gian của sóng ánh sáng gần bề mặt của vật nóng phát ra nó bị hạn chế bởi giá trị $\ab{\rho}{coh}$ đối với một vài bước sóng.
Khi khoảng cách từ nguồn tăng lên, mức độ nhất quán không gian tăng lên.
Bức xạ của tia laser\footnote{Laser sẽ được đề cập trong Tập III.} có sự nhất quán không gian và thời gian rất lớn.
Tại lỗ thoát của tia laser, sự nhất quán không gian được quan sát thấy trên toàn bộ mặt cắt ngang của chùm sáng.

% It would seem possible to observe interference by passing light propagating from an arbitrary source through two slits in an opaque screen.
% With a small spatial coherence of the wave falling on the slits, however, the beams of light passing through them will be incoherent, and an interference pattern will be absent.
% The English scientist Thomas Young (1773-1829) in 1802 obtained interference from two slits by increasing the spatial coherence of the light falling on the slits.
% Young achieved such an increase by first passing the light through a small aperture in an opaque screen.
% This light was used to illuminate the slits in a second opaque screen.
% Thus, for the first time in history, Young observed the interference of light waves and determined the lengths of these waves.
Có thể quan sát được hiện tượng giao thoa bằng cách phát xạ ánh sáng từ một nguồn bất kỳ qua hai khe hở trên một màn chắn mờ.
Tuy nhiên, chỉ với một sự nhất quán không gian nhỏ của sóng chiếu vào các khe hở, các chùm ánh sáng đi qua chúng sẽ không kết hợp và sẽ không có sự giao thoa.
Nhà khoa học người Anh Thomas Young (1773-1829) vào năm 1802 đã thu được hiện tượng giao thoa từ hai khe hở bằng cách tăng sự nhất quán không gian của ánh sáng chiếu vào các khe hở.
Young đã đạt được như vậy bằng cách truyền ánh sáng qua một lỗ nhỏ trên một màn chắn mờ.
Ánh sáng này được sử dụng để chiếu sáng các khe hở trên một màn chắn mờ thứ hai.
Do đó, lần đầu tiên trong lịch sử, Young đã quan sát được hiện tượng giao thoa của sóng ánh sáng và xác định được độ dài của các sóng này.

% \section{Ways of Observing the Interference of Light}\label{sec:17_3}
\section{Cách để quan sát giao thoa ánh sáng}\label{sec:17_3}

% Let us consider two concrete interference layouts of which one uses reflection for splitting a light wave into two parts, and the other refraction of light.
Chúng ta hãy xem xét hai cách bố trí giao thoa trong đó một cách sử dụng sự phản xạ để chia sóng ánh sáng thành hai phần và cách còn lại sử dụng sự khúc xạ ánh sáng.

% \textbf{Fresnel's Double Mirror.}
% Two plane contacting mirrors $0$M and $0$N are arranged so that their reflecting surfaces form an obtuse angle close to $\pi$ (\fig{17_8}).
% Hence, the angle $\varphi$ in the figure is very small.
% A straight light source S (for example, a narrow luminous slit) is placed parallel to the line of intersection of the mirrors $0$ (perpendicular
% to the plane of the drawing) at a distance $r$ from it.
% The mirrors reflect two cylindrical coherent waves onto screen Sc.
% They propagate as if they were emitted by virtual sources S$_1$ and S$_2$.
% Opaque screen Sc$_1$ prevents the direct propagation of the light from source S to screen Sc.
\textbf{Gương đôi Fresnel.}
Hai gương phẳng tiếp xúc $0$M và $0$N được sắp xếp sao cho bề mặt phản xạ của chúng tạo thành một góc tù cận $\pi$ (\fig{17_8}).
Do đó, góc $\varphi$ trong hình rất nhỏ.
Một nguồn sáng thẳng S (ví dụ, một khe sáng hẹp) được đặt song song với đường giao nhau của các gương $0$ (vuông góc
với mặt phẳng của bản vẽ) ở khoảng cách $r$.
Các gương phản chiếu hai sóng kết hợp hình trụ lên màn hình Sc.
Các sóng lan truyền như thể chúng được phát ra từ các nguồn ảo S$_1$ và S$_2$.
Màn hình mờ đục Sc$_1$ ngăn cản sự lan truyền trực tiếp của ánh sáng từ nguồn S đến màn hình Sc.

\begin{figure}[!htb]
	\begin{center}
		\includegraphics[scale=0.98]{figures/ch_17/fig_17_8.pdf}
		\caption[]{}
        % \caption[]{Fresnel's double mirror experiment. A light source S is placed parallel to the line of intersection of the mirrors $0$, reflecting two cylindrical coherent waves onto screen Sc. Opaque screen Sc$_1$ prevents the direct propagation of the light from source S to screen Sc.}
		\label{fig:17_8}
	\end{center}
	\vspace{-0.9cm}
\end{figure}

% Ray $0$Q is the reflection of ray S$0$ from mirror $0$M, and ray $0$P is the reflection of ray S$0$ from mirror $0$N.
% It is easy to see that the angle between rays $0$P and $0$Q is $2\varphi$.
% Since S and S$_1$ are symmetrical relative to $0$M, the length of segment $0$S$_1$ equals $0$S, \ie, $r$.
% Similar reasoning leads to the same result for segment $0$S$_2$.
% Thus, the distance between sources S$_1$ and S$_2$ is
Tia $0$Q là tia phản xạ của tia S$0$ từ gương $0$M, và tia $0$P là tia phản xạ của tia S$0$ từ gương $0$N.
Dễ dàng thấy rằng góc giữa các tia $0$P và $0$Q là $2\varphi$.
Vì S và S$_1$ đối xứng với $0$M, nên độ dài của đoạn $0$S$_1$ bằng $0$S, \ie, $r$.
Suy luận tương tự dẫn đến kết quả tương tự cho đoạn $0$S$_2$.
Do đó, khoảng cách giữa các nguồn S$_1$ và S$_2$ là
\begin{equation*}
    d = 2r \sin\varphi \approx 2 r \varphi.
\end{equation*}

\noindent
% Inspection of \fig{17_8} shows that $a=r \cos\varphi \approx r$. Hence,
Xem xét \fig{17_8} ta thấy rằng $a=r \cos\varphi \approx r$. Do đó,
\begin{equation*}
    l = r + b,
\end{equation*}

\noindent
% where $b$ is the distance from the line of intersection of the mirrors $0$ to screen Sc.
Với $b$ là khoảng cách giữa giao điểm cảu hai gương $0$ đến màn Sc.

% Using the values of $d$ and $l$ we have found in \eqn{17_10}, we obtain the width of an interference fringe:
Sử dụng giá trị $d$ và $l$ tìm được trong \eqn{17_10}, ta tìm được chiều rộng của vân sáng:
\begin{equation}\label{eq:17_28}
    \Delta{x} = \frac{r + b}{2 r \varphi} \lambda.
\end{equation}

\noindent
% The region of wave overlapping PQ has a length of $2b\tan\varphi\approx 2b\varphi$.
% Dividing this length by the width of a fringe $\Delta{x}$, we find the maximum number of interference fringes that can be observed with the aid of Fresnel's double mirror at the given parameters of a layout:
Vùng sóng chồng chất PQ có chiều dài là $2b\tan\varphi\approx 2b\varphi$.
Chia chiều dài này cho chiều rộng của một vân $\Delta{x}$, ta tìm được số vân giao thoa tối đa có thể quan sát được với
sự trợ giúp của gương đôi Fresnel tại các tham số đã cho:
\begin{equation}\label{eq:17_29}
    N = \frac{4 b r \varphi^2}{\lambda (r + b)}.
\end{equation}

\noindent
% For all these fringes to be visible indeed, it is essential that $N/2$ be not greater than $\ab{m}{extr}$ determined by expression \eqref{eq:17_22}.
Để nhìn thấy tất cả các vân, $N/2$ phải bé hơn $\ab{m}{extr}$ được xác định bởi biểu thức \eqref{eq:17_22}.

\textbf{Lăng kính kép Fresnel}.
% Two prisms with a small refracting angle $\theta$ made from a single piece of glass have one common face (\fig{17_9}).
% A straight light source S is arranged parallel to this face at a distance a from it.
Hai lăng kính với góc khúc xạ nhỏ $\theta$ được làm từ một tấm kính có một mặt chung (\fig{17_9}).
Một nguồn sáng thẳng S được đặt song song với mặt này tại một khoảng cách nhất định.

\begin{figure}[!htb]
	\begin{center}
		\includegraphics[scale=0.98]{figures/ch_17/fig_17_9.pdf}
		\caption[]{}
        % \caption[]{Fresnel's biprism experiment. Two prisms with a small refracting angle $\theta$ made from a single piece of glass have one common face. A straight light source S is arranged parallel to this face at a distance a from it.}
		\label{fig:17_9}
	\end{center}
	\vspace{-0.9cm}
\end{figure}

% It can be shown that when the refracting angle a of the prism is very small and the angles of incidence of the rays on the face of the prism are not very great, all the rays are deflected by the prism through a practically identical angle equal to
Có thể chứng minh rằng khi góc khúc xạ a của lăng kính rất nhỏ và góc tới của các tia sáng trên mặt lăng kính không quá lớn thì tất cả các tia sáng đều bị lăng kính làm lệch đi một góc gần bằng nhau bằng
\begin{equation*}
    \varphi = (n - 1) \theta
\end{equation*}

\noindent
% ($n$ is the refractive index of the prism).
% The angle of incidence of the rays on the biprism is not great.
% Therefore, all the rays are deflected by each half of the biprism through the same angle.
% As a result, two coherent cylindrical waves are formed emerging from virtual sources S$_1$ and S$_2$ in the same plane as S.
% The distance between the sources is
($n$ là chiết suất của lăng kính).
Góc tới của các tia sáng trên lăng kính kép không lớn.
Do đó, tất cả các tia sáng đều bị một nửa lăng kính kép lệch đi một góc bằng nhau.
Kết quả là, hai sóng hình trụ kết hợp xuất phát từ các nguồn ảo S$_1$ và S$_2$ trong cùng một mặt phẳng với S.
Khoảng cách giữa các nguồn là
\begin{equation*}
    d = 2a \sin\varphi \approx 2a\varphi = 2a (n - 1) \theta.
\end{equation*}

\noindent
% The distance from the sources to the screen is
Khoảng cách giữa các nguồn là
\begin{equation*}
    l = a + b.
\end{equation*}

% We find the width of an interference fringe by \eqn{17_10}:
Ta tìm được chiều rộng của vân giao thoa bằng \eqn{17_10}:
\begin{equation}\label{eq:17_30}
    \Delta{x} = \frac{(a + b)}{2 a (n - 1) \theta} \lambda.
\end{equation}

\noindent
% The region of overlapping of the waves PQ has the length
Vùng sóng chồng chập PQ có độ dài
\begin{equation*}
    2b \tan\varphi \approx 2b\varphi = 2b (n - 1) \theta.
\end{equation*}

\noindent
% The maximum number of fringes observed is
Số vân cực đại quan sát được là
\begin{equation}\label{eq:17_31}
    N = \frac{4ab (n-1)^2 \theta^2}{\lambda (a+b)}.
\end{equation}

% \section{Interference of Light Reflected from Thin Plates}\label{sec:17_4}
\section{Giao thoa của ánh sáng phản xạ từ các tấm kính mỏng}\label{sec:17_4}

% When a light wave falls on a thin transparent plate (or film), reflection occurs from both surfaces of the plate.
% The result is the production of two light waves that in known conditions can interfere.
Khi ánh sáng chiếu vào một tấm kính mỏng trong suốt (hoặc phim), sự phản xạ xảy ra từ cả hai bề mặt của tấm.
Kết quả tạo ra hai sóng ánh sáng có thể giao thoa trong điều kiện cho trước.

% Assume that a plane light wave which can be considered as a parallel beam of rays falls on a transparent plane-parallel plate (\fig{17_10}).
% The plate reflects upward two parallel beams of light.
% One of them was formed as a result of reflection from the top surface of the plate and the second as a result of reflection from its bottom surface (in \fig{17_10} each of these beams is  represented by only one ray).
% The second beam is refracted when it enters the plate and leaves it.
% In addition to these two beams, the plate throws upward beams produced as a result of three-, five-fold, etc. reflection from the plate surfaces.
% Owing to their small intensity, however, we shall take no account of these beams\footnote{At $n=1.5$, about $5\%$ of the incident luminous flux is reflected from the surface of the plate (see the last paragraph of \sect{16_3}). After two reflections, the intensity will be $0.05\times 0.05$ or $0.25\%$ of the intensity of the initial beam. After three reflections, the relevant figure is $0.05\times 0.05\times 0.05$, or $0.0125\%$, which is $1/400$ of the intensity of the singly reflected beam.}.
% We shall also display no interest in the beams passing through the plate.
Giả sử rằng một sóng ánh sáng phẳng có thể được coi là một chùm tia song song chiếu vào một tấm kính phẳng song song trong suốt (\fig{17_10}).
Đĩa phản xạ hai chùm ánh sáng song song hướng lên.
Một trong số chúng được hình thành do phản xạ từ mặt trên cùng và chùm thứ hai là do phản xạ từ mặt dưới cùng (trong \fig{17_10} mỗi chùm tia này chỉ được biểu diễn bằng một tia).
Chùm tia thứ hai bị khúc xạ khi nó đi vào kính và rời khỏi kính.
Ngoài hai chùm tia này, các chùm tia phản xạ ba, năm lần,... đều phản xạ hướng lên trên.
Tuy nhiên, do cường độ của chúng nhỏ nên chúng ta sẽ không tính đến các chùm tia này\footnote{Tại $n=1.5$, khoảng $5\%$ thông lượng sáng tới sẽ được phản xạ từ bề mặt tấm (xem đoạn cuối của \sect{16_3}). Sau hai lần phản xạ, cường độ sẽ là $0,05\times 0,05$ hoặc $0,25\%$ cường độ của chùm tia ban đầu. Sau ba lần phản xạ, con số liên quan là $0,05\times 0,05\times 0,05$ hoặc $0,0125\%$, bằng $1/400$ cường độ của chùm tia phản xạ đơn lẻ.}.
Chúng ta cũng sẽ không quan tâm đến các chùm tia đi qua tấm.

\begin{figure}[!htb]
	\begin{center}
		\includegraphics[scale=0.95]{figures/ch_17/fig_17_10.pdf}
		\caption[]{}
        % \caption[]{Interference of light in a parallel-plate.}
		\label{fig:17_10}
	\end{center}
	\vspace{-0.8cm}
\end{figure}

% The path difference acquired by rays $1$ and $2$ before they meet at point C is
Hiệu đường đi của các tia sáng $1$ và $2$ trước khi giao nhau tại C là
\begin{equation}\label{eq:17_32}
    \Delta = ns_2 - s_1,
\end{equation}

\noindent
% where $s_1$ is the length of segment BC, $s_2$ is the total length of segments AO and OC and $n$ the refractive index of the plate.
với $s_1$ là độ dài BC, $s_2$ là tổng độ dài AO và OC, và $n$ là chiết suất của tấm kính.

% We assume that the refractive index of the medium surrounding the plate is unity.
% A glance at \fig{17_10} shows that $s_1 = 2b \tan\theta_2 \times \sin\theta_1$, and $s_2=2b/\cos\theta_2$ (here, $b$ is the thickness of the plate).
Ta giả sử rằng chết suất môi trường xung quanh tấm kính là một.
\fig{17_10} cho thấy $s_1 = 2b \tan\theta_2 \times \sin\theta_1$ và $s_2=2b/\cos\theta_2$ (với $b$ là độ dày tấm kính).

% Using these values in \eqn{17_32}, we get
Sử dụng các giá trị từ \eqn{17_32}, ta được
\begin{equation*}
    \Delta = \frac{2bn}{\cos\theta_2} - 2b\tan\theta_2\sin\theta_1 = 2b \frac{n^2 - n \sin\theta_2 \sin\theta_1}{n \cos\theta_2}.
\end{equation*}

\noindent
% Substituting $\sin\theta_1$ for $n\sin\theta_1$ and taking into account that
Thay $\sin\theta_1$ cho $n\sin\theta_1$ và tính đến việc
\begin{equation*}
    n \cos\theta_2 = \sqrt{n^2 - n^2 \sin^2\theta_2} = \sqrt{n^2 - \sin^2\theta_1},
\end{equation*}

\noindent
% it is easy to give the equation for $\Delta$ the form
ta dễ dàng có được phương trình của $\Delta$ dưới dang
\begin{equation}\label{eq:17_33}
    \Delta = 2b \sqrt{n^2 - \sin^2\theta_1}.
\end{equation}

% When calculating the phase difference $\delta$ between the oscillations in rays $1$ and $2$, it is necessary, in addition to the optical path difference $\Delta$, to take into account the possibility of a change in the phase of the wave upon reflection (see \sect{16_3}).
% At point A (see \fig{17_10}), reflection occurs from the interface between the optically less dense medium and the optically denser one.
% Consequently, the wave phase experiences a change by $\pi$.
% At point $0$, reflection occurs from the interface between the optically denser medium and the optically less dense one, so that there is no jump in the phase.
% Hence, an additional phase difference equal to $\pi$ is produced between rays $1$ and $2$.
% It can be taken into account by adding to $\Delta$ (or subtracting from it) half a wavelength in a vacuum. The result is
Khi tính độ lệch pha $\delta$ giữa các dao động trong tia $1$ và $2$, ngoài độ lệch quang trình $\Delta$, cần phải tính đến khả năng thay đổi pha của sóng khi phản xạ (xem \sect{16_3}).
Tại điểm A (xem \fig{17_10}), phản xạ xảy ra tại mặt tiếp xúc giữa môi trường có chiết suất thấp hơn và môi trường có chiết suất cao hơn.
Do đó, pha sóng biến thiên một lượng $\pi$.
Tại điểm $0$, phản xạ xảy ra từ mặt tiếp xúc giữa môi trường có chiết suất cao và môi trường có chiết suất thấp, do đó không có sự nhảy pha.
Do đó, độ lệch pha bổ sung bằng $\pi$ được tạo ra giữa các tia $1$ và $2$.
Có thể tính đến điều này bằng cách thêm (hoặc trừ đi) một nửa bước sóng trong chân không vào $\Delta$. Kết quả là
\begin{equation}\label{eq:17_34}
    \Delta = 2 b \sqrt{n^2 - \sin^2\theta_1} - \frac{\lambda_0}{2}.
\end{equation}

% Thus, when a plane wave falls on the plate, two reflected waves are formed, and their path difference is determined by \eqn{17_34}.
% Let us determine the conditions in which these waves will be coherent and can interfere.
% We shall consider two cases.
Do đó, khi một sóng chiếu lên tấm kính, hai sóng phản xạ được hình thành và hiệu số đường đi của chúng được xác định bởi \eqn{17_34}.
Bây giờ ta xác định các điều kiện mà các sóng này sẽ kết hợp và có thể giao thoa.
Chúng ta sẽ xem xét hai trường hợp.

% \textbf{1. A Plane-Parallel Plate.}
% Both plane reflected waves propagate in one direction making an angle equal to the angle of incidence $\theta_1$ with a normal to the plate.
% These waves can interfere if conditions of both temporal and spatial coherence are observed.
\textbf{1. Tấm kính phẳng song song}
Hai sóng phản xạ truyền theo một hướng và tạo một góc bằng với góc tới $\theta_1$ vuông góc với tấm kính.
Các sóng này có thể giao thoa nếu các điều kiện của cả nhất quán thời gian và không gian được quan sát.

% For temporal coherence to take place, the path difference given by \eqn{17_34} must not exceed the coherence length equal to $\lambda^2/\Delta{\lambda}\approx\lambda_0^2/\Delta{\lambda_0}$ [see expression \eqref{eq:17_21}].
% Consequently, the condition
Để nhất quán không gian xảy ra, hiệu đường đi được cho bởi \eqn{17_34} không được vượt quá độ dài kết hợp $\lambda^2/\Delta{\lambda}\approx\lambda_0^2/\Delta{\lambda_0}$ [xem biểu thức \eqref{eq:17_21}].
Hệ quả là điều kiện
\begin{equation*}
    2b \sqrt{n^2 - \sin^2\theta_1} - \frac{\lambda_0}{2} < \frac{\lambda_0^2}{\Delta{\lambda_0}},
\end{equation*}

\noindent
hoặc
\begin{equation*}
    b < \frac{\lambda_0 (\lambda_0/\Delta{\lambda_0} + 1/2)}{2 \sqrt{n^2 - \sin^2\theta_1}},
\end{equation*}

\noindent
% must be observed.
% In the obtained relation, we may disregard $1/2$
% in comparison with $\lambda_0/\Delta{\lambda_0}$.
% The expression $\sqrt{n^2-\sin^2\theta_1}$ has a magnitude of the order of unity\footnote{For $n=1.5$, the magnitude of this expression varies within the limits from $1.12$ (at $\theta_1=\pi/2$) to $1.5$ (at $\theta_1=0$).}.
% We can therefore write
phải được quan sát.
Trong mối quan hệ thu được, chúng ta có thể bỏ qua $1/2$
khi so sánh với $\lambda_0/\Delta{\lambda_0}$.
Biểu thức $\sqrt{n^2-\sin^2\theta_1}$ có độ lớn đồng nhất\footnote{Với $n=1.5$, độ lớn của biểu thức này thay đổi trong giới hạn từ $1.12$ (tại $\theta_1=\pi/2$) đến $1.5$ (tại $\theta_1=0$).}.
Do đó, chúng ta có thể viết
\begin{equation}\label{eq:17_35}
    b < \frac{\lambda_0^2}{2 \Delta{\lambda_0}}
\end{equation}

\noindent
% (the double plate thickness must be less than the coherence length).
(độ dày của kính đôi phải bé hơn độ dài kết hợp). 

% Thus, the reflected waves will be coherent only if the plate thickness $b$ does not exceed the value determined by expression \eqref{eq:17_35}.
% Assuming that $\lambda_0=\SI{5000}{\angstrom}$ and $\Delta{\lambda_0}=\SI{20}{\angstrom}$, we get the extreme value of the thickness equal to
Do đó, sóng phản xạ sẽ chỉ kết hợp nếu độ dày của tấm $b$ không vượt quá giá trị được xác định bởi biểu thức \eqref{eq:17_35}.
Giả sử rằng $\lambda_0=\SI{5000}{\angstrom}$ và $\Delta{\lambda_0}=\SI{20}{\angstrom}$, chúng ta sẽ có giá trị cực đại của độ dày bằng
\begin{equation}\label{eq:17_36}
    \frac{5000^2}{2 \times 20} \approx \SI{6e5}{\angstrom} = \SI{0.06}{\milli\metre}.
\end{equation}

Now, let us consider the conditions for observance of spatial coherence.
Let us place screen Sc in the path of the reflected beams (\fig{17_11}).
Rays $1'$ and $2'$ arriving at point P$'$ will be at a distance $\rho'$ apart in the incident beam.
If this distance does not exceed the coherence radius $\ab{\rho}{coh}$ of the incident wave, rays $1'$ and $2'$ will be coherent and will produce at point P$'$ an illumination determined by the value of the path difference $\Delta$ corresponding to the angle of incidence $\theta_1$.
The other pairs of rays travelling at the same angle $\theta_1'$ will produce the same illumination at the other points of the screen.
The screen will thus be uniformly illuminated (in the particular case when $\Delta=(n+1/2)\lambda_0$, the screen will be dark).
When the inclination of the beam is changed (\ie, when the angle $\theta_1$ is changed), the illumination of the screen will change too.

\begin{figure}[!htb]
	\begin{center}
		\includegraphics[scale=1]{figures/ch_17/fig_17_11.pdf}
		\caption[]{}
        % \caption[]{Spatial coherence in a plane-parallel plate. Rays $1'$ and $2'$ arriving at point P$'$ will be at a distance $\rho'$ apart in the incident beam. If $\rho'<\ab{\rho}{coh}$ of the incident wave, rays $1'$ and $2'$ will be coherent and will produce at point P$'$ an illumination determined by the value of the path difference $\Delta$ corresponding to the angle of incidence $\theta_1$.}
		\label{fig:17_11}
	\end{center}
	\vspace{-0.8cm}
\end{figure}

A glance at \fig{17_10} shows that the distance between the incident rays $1$ and $2$ is
\begin{equation}\label{eq:17_37}
    \rho = 2b\tan\theta_2\sin\theta_1 = \frac{b \sin(2\theta_1)}{\sqrt{n^2 - \sin^2\theta_1}}.
\end{equation}

\noindent
If we assume that $n=1.5$, then, for $\theta_1=\SI{45}{\degree}$ we get $\rho=0.8b$, and for $\theta_1=\SI{10}{\degree}$ we get $\rho=0.1b$.
For normal incidence ($\theta_1=0$), we have $\rho=0$ at any $n$.

The coherence radius of sunlight has a value of the order of \SI{0.05}{\milli\metre} [see \eqn{17_27}].
At an angle of incidence of \SI{45}{\degree}, we may assume that $\rho\approx b$.
Hence, for interference to occur in these conditions, the relation
\begin{equation}\label{eq:17_38}
    b < \SI{0.05}{\milli\metre}
\end{equation}

\noindent
must be observed [compare with \eqn{17_36}].
For an angle of incidence of about \SI{10}{\degree}, spatial coherence will be retained at a plate thickness not exceeding \SI{0.5}{\milli\metre}.
We thus arrive at the conclusion that owing to the restrictions imposed by temporal and spatial coherence, interference is observed when a plate is illuminated by sunlight only if the thickness of the plate does not exceed a few hundredths of a millimetre.
Upon illumination with light having a greater degree of coherence, interference is also observed in reflection from thicker plates or films.

Interference from a plane-parallel plate is observed in practice by placing in the path of the reflected beams a lens that gathers the rays at one of the points of the screen in the focal plane of the lens (\fig{17_12}).
The illumination at this point depends on the value of quantity \eqref{eq:17_34}.
When $\Delta=m\lambda_0$, we get maxima, and when $\Delta=(m+1/2)\lambda_0$---minima of the intensity ($m$ is an integer).
The condition for the maximum intensity has the form
\begin{equation}\label{eq:17_39}
    2b \sqrt{n^2 - \sin^2\theta_1} = \parenthesis{m + \frac{1}{2}} \lambda_0.
\end{equation}

\begin{figure}[!htb]
	\begin{center}
		\includegraphics[scale=1]{figures/ch_17/fig_17_12.pdf}
		\caption[]{}
        % \caption[]{Interference from a plane-parallel plate observed by placing in the path of the reflected beams a lens that gathers the rays at one of the points of the screen in the focal plane of the lens.}
		\label{fig:17_12}
	\end{center}
	\vspace{-0.8cm}
\end{figure}

Assume that a thin plane-parallel plate is illuminated by diffuse monochromatic light (see \fig{17_12}).
Let us arrange a lens parallel to the plate and put a screen in the focal plane of the lens.
Diffuse light contains rays of the most diverse directions.
The rays parallel to the plane of the drawing and falling on the plate at the angle $\theta_1'$ after reflection from both surfaces of the plate will be gathered by the lens at point P$'$ and will set up at this point an illumination determined by the value of the optical path difference.
Rays propagating in other planes but falling on the plate at the same angle $\theta_1'$
will be gathered by the lens at other points at the same distance as point P$'$ from centre $0$ of the screen.
The illumination at all these points will be the same.
Thus, the rays falling on the plate at the same angle $\theta_1'$ will produce on the screen a collection of identically illuminated points arranged along a circle with its centre at $0$.
Similarly, the rays falling at a different angle $\theta_1''$ will produce on the screen a collection of identically (but different in value because $\Delta$ is different) illuminated points arranged along a circle of another radius.
The result will be the appearance on the screen of a system of alternating bright and dark circular fringes with a common centre at point $0$.
Each fringe is formed by the rays falling on the plate at the same angle $\theta_1$.
This is why interference fringes produced in such conditions are known as fringes of equal inclination.
When the lens is arranged differently relative to the plate (the screen must coincide with the focal plane of the lens in all cases), the fringes of equal inclination will have another shape.

Every point of an interference pattern is due to rays which formed a parallel beam before passing through the lens.
Hence, in observing fringes of equal inclination, the screen must be placed in the focal plane of the lens, \ie, in the same way in which it is arranged to produce an image of infinitely remote objects on it.
Accordingly, fringes of equal inclination are said to be localized at infinity.
The part of the lens can be played by the crystalline lens, and that of the screen by the retina of the eye.
In this case for observing fringes of equal inclination, the eye must be accommodated as when looking at very remote objects.

According to \eqn{17_39}, the position of the maxima depends on the wavelength $\lambda_0$.
Therefore, in white light, we get a collection of
fringes displaced relative to one another and formed by rays of different colours; the interference pattern acquires the colouring of a rainbow.
The possibility of observing an interference pattern in white light is determined by the ability of the eye to distinguish light tints of close wavelengths.
The average human eye perceives rays differing in wavelength by less than \SI{20}{\angstrom} as having the same colour.
Therefore, to assess the conditions in which interference from plates can be observed in white light, we must assume that $\Delta{\lambda_0}$ equals \SI{20}{\angstrom}.
We took exactly this value in assessing the thickness of a plate [see \eqn{17_36}].

\textbf{2. Plate of Varying Thickness.}
Let us take a plate in the form of a wedge with an apex angle of $\varphi$ (\fig{17_13}).
Assume that a parallel beam of rays falls on it.
Now the rays reflected from different surfaces of the plate will not be parallel.
Two rays that practically merge before falling on the plate (in \fig{17_13} they are depicted in the form of a single straight line designated by the figure $1'$) intersect after reflection at point $Q'$.
The two rays $1''$ practically merging intersect
at point Q$''$ after reflection.
It can be shown that points Q$'$, Q$''$ and other points similar to them lie in one plane passing through apex $0$ of the wedge.
Ray $1'$ reflected from the bottom surface of the wedge and ray $2'$ reflected from its top surface will intersect at point R$'$ that is closer to the wedge than Q$'$.
Similar rays $1'$ and $3'$ will intersect at point P$'$ that is farther from the wedge surface than Q$'$.

\begin{figure}[!htb]
	\begin{center}
		\includegraphics[scale=1]{figures/ch_17/fig_17_13.pdf}
		\caption[]{}
        % \caption[]{Interference of rays in a plate of varying thickness (wedge).}
		\label{fig:17_13}
	\end{center}
	\vspace{-0.8cm}
\end{figure}

The directions of propagation of the waves reflected from the top and bottom surfaces of the wedge do not coincide.
Temporal coherence will be observed only for the parts of the waves reflected from places of the wedge for which the thickness satisfies condition \eqref{eq:17_35}.
Assume that this condition is observed for the entire wedge.
In addition, assume that the coherence radius is much greater than the wedge length.
Hence, the reflected waves will be coherent in the entire space over the wedge, and no matter at what distance from the wedge the screen is, an interference pattern will be observed on it in the form of fringes parallel to the wedge apex $0$ (see the last three paragraphs of \sect{17_1}).
This, particularly, is how matters are when a wedge is illuminated by light emitted by a laser.

With restricted spatial coherence, the region of localization of the interference pattern (\ie, the region of space in which an interference pattern can be seen on a screen placed in it) will be restricted too.
If we arrange a screen so that it pass.3s through points Q$'$, Q$''$, $\ldots$ (see screen Sc in \fig{17_13}), an interference pattern will appear on it even if the spatial coherence of the falling wave is extremely small (rays that coincided before falling on the wedge will intersect at points on the screen).
At a small wedge angle $\varphi$, the path difference of the rays can be calculated with sufficient accuracy by \eqn{17_34} taking as $b$ the thickness of the plate at the place where the rays fall on it.
Since the path difference for the rays reflected from different sections of the wedge is now different, the illumination of the screen will be non-uniform---bright and dark fringes will appear on it (see the dash curve showing the illumination of screen Sc in \fig{17_13}).
Each of these fringes is produced as a result of reflection from sections of the wedge having the same thickness.
This is why they are known as \textbf{fringes of equal thickness}.

Upon displacement of the screen from position Sc in a direction away from the wedge or toward it, the degree of spatial coherence of the incident wave begins to tell.
If in the position of the screen denoted in \fig{17_13} by Sc$'$, the distance $\rho'$ between the incident rays $1'$ and $2'$ becomes of the order of the coherence radius, no interference pattern will be observed on screen Sc$'$.
Similarly, the pattern vanishes when the screen is at position Sc$''$.

Thus, the interference pattern produced when a plane wave is reflected from a wedge is localized in a certain region near the surface of the wedge.
This region becomes narrower when the degree of spatial coherence of the incident wave diminishes.
Inspection of \fig{17_13} shows that the conditions for both temporal and spatial coherence
become more favourable nearer to the apex of the wedge.
Therefore, the distinctness of the interference pattern diminishes when moving from the apex of the wedge to its base.
A pattern may be observed only for the thinner part of the wedge.
For its remaining part, the screen will be uniformly illuminated.

\begin{figure}[!htb]
	\begin{center}
		\includegraphics[scale=0.9]{figures/ch_17/fig_17_14.pdf}
		\caption[]{}
        % \caption[]{Fringes of equal thickness are observed by placing a lens near a wedge, and a screen behind the lens.}
		\label{fig:17_14}
	\end{center}
	\vspace{-0.8cm}
\end{figure}

Practically, fringes of equal thickness are observed by placing a lens near a wedge, and a screen behind the lens (\fig{17_14}).
The part of the lens can be played by the crystalline lens, and of the screen by the retina of the eye.
If the screen behind the lens is in a plane conjugated with the plane designated by Sc in \fig{17_13} (the eye is accordingly accommodated to this plane), the pattern will be most distinct.
When the screen onto which the image is projected is moved (or when the lens is moved), the pattern will become less distinct and will vanish completely if the plane conjugated with the screen passes beyond the limits of the region of localization of the interference pattern observed without a lens.

When observed in white light, the fringes will be coloured, so that the surface of a plate or film will have rainbow colouring.
For example, thin films of oil on the surface of water and soap films have such colouring.
The temper colours appearing on the surface of steel articles when they are hardened are also due to interference from a film of transparent oxides.

Let us compare the two cases of interference upon reflection from thin films which we have considered.
Fringes of equal inclination are obtained when a
plate of constant thickness ($b=\text{constant}$) is illuminated by diffuse light containing
rays of various directions ($\theta_1$ is varied within more or less broad limits).
Fringes of equal inclination are localized at infinity.
Fringes of equal thickness are observed when a plate of varying thickness ($b$ varies) is illuminated by a parallel beam of light ($\theta_1=\text{constant}$).
Fringes of equal thickness are localized near the plate.
In real conditions, for example, when observing rainbow colours on a soap or oil film, both the angle of incidence of the rays and the thickness of the film are varied.
In this case, fringes of a mixed type are observed.

We must note that interference from thin films can be observed not only in reflected, but also in transmitted light.

\textbf{Newton's Rings.}
A classical example of fringes of equal thickness
are \textbf{Newton's rings}.
They are observed when light is reflected from a thick plane-parallel glass plate in contact with a plano-convex lens having a large radius of curvature (\fig{17_15}).
The part of a thin film from whose surfaces coherent waves are reflected is played by the air gap between the plate and the lens (owing to the great thickness of the plate and the lens, no interference fringes appear as a result of reflections from other surfaces).
With normal incidence of the light, fringes of equal thickness have the form of concentric rings, and with inclined incidence, of ellipses.
Let us find the radii of Newton's rings produced when light falls along a normal to the plate.
In this case, $\sin\theta_1=0$, and the optical path difference equals the double thickness of the gap [see \eqn{17_33}, it is assumed that $n=1$ in the gap].
It follows from \fig{17_15} that
\begin{equation}\label{eq:17_40}
    R^2 = (R - b)^2 + r^2 \approx R^2 - 2Rb + r^2,
\end{equation}

\noindent
where $R$ is the radius of curvature of the lens, $r$ is the radius of a circle with the identical gap $b$ corresponding to all of its points.

\begin{figure}[!htb]
	\begin{center}
		\includegraphics[scale=1]{figures/ch_17/fig_17_15.pdf}
		\caption[]{}
        % \caption[]{Newton's rings are observed when light is reflected from a thick plane-parallel glass plate in contact with a plano-convex lens having a large radius of curvature.}
		\label{fig:17_15}
	\end{center}
	\vspace{-0.8cm}
\end{figure}

Owing to the smallness of $b$, in expression \eqref{eq:17_40} we have disregarded the quantity $b^2$ in comparison with $2Rb$.
In accordance with expression \eqref{eq:17_40}, $b=r^2/(2R)$.
To take account of the change in the phase by $\pi$ occurring upon reflection from the plate, we must add $\lambda_0/2$ to $2b=r^2/R$.
The result is
\begin{equation}\label{eq:17_41}
    \Delta = \frac{r^2}{R} + \frac{\lambda_0}{2}.
\end{equation}

At points for which $\Delta=m'\lambda_0 = 2m'(\lambda_0/2)$, maxima appear, and at points for which $\Delta=(m'+1/2) \lambda_0 = (2m'+1)(\lambda_0/2)$, minima of the intensity appear.
Both conditions can be combined into the single one
\begin{equation*}
    \Delta = m \frac{\lambda_0}{2}
\end{equation*}

\noindent
maxima corresponding to even values of $m$, and minima of the intensity, to odd values.
Introducing into this expression \eqn{17_41} for $\Delta$ and solving the resulting equation relative to $r$, we find the radii of bright and dark Newton's rings:
\begin{equation}\label{eq:17_42}
    r = \parenthesis{\frac{R \lambda_0 (m-1)}{2}}^{1/2}\quad (m = 1, 2, 3, \ldots).
\end{equation}

\noindent
Radii of bright rings correspond to even $m$'s, and radii of dark rings to odd ones.
The value $r=0$ corresponds to $m=1$, \ie, to the point at the place of contact of the plate and the lens.
A minimum of intensity is observed at this point.
It is due to the change in the phase by $\pi$ when a light wave is reflected from the plate.

\textbf{Coating of Lenses.}
The coating of lenses is based on the interference of light when reflected from thin films.
The transmission of light through each refracting surface of a lens is attended by the reflection of about four per cent of the incident light.
In multicomponent lenses, such reflections occur many times, and the total loss of the light flux reaches an appreciable value.
In addition, the reflections from the lens surfaces result in the appearance of highlights.
The reflection of light is eliminated by applying a thin film of a substance having a refractive index other than that of the lens to each free surface of the latter.
The components obtained in this way are called \textbf{coated lenses}.
The thickness of the coating is chosen so that the waves reflected from both its surfaces interfere destructively.
An especially good result is obtained if the refractive index of the film equals the square root of the refractive index of the lens.
When this condition is satisfied, the intensity of both waves reflected from the film surfaces is the same.

\section{The Michelson Interferometer}\label{sec:17_5}

Many varieties of interference instruments called \textbf{interferometers} are in use.
Figure \ref{fig:17_16} is a schematic view of a \textbf{Michelson interferometer}\footnote{Named after its inventor, the American physicist Albert Michelson (1852-1931).}.
A light beam from source S falls on semitransparent plate P$_1$ coated with a thin layer of silver (this layer is depicted by dots in the figure).
Half of the incident light flux is reflected by plate P$_1$ in the direction of ray $1$ and half passes through the plate and propagates in the direction of ray $2$. Beam $1$ is reflected from mirror M$_1$ and returns to P$_1$, where it is split into two beams of equal intensity.
One of them passes through the plate and forms beam $1'$, and the second one is reflected in the direction of S.
The latter beam will no longer interest us.
Beam $2$ after being reflected by mirror M$_2$ also returns to plate P$_1$ where it is divided into two parts: beam $2'$ reflected from the semitransparent layer, and the beam transmitted through the layer, which will also no longer interest us.
Light beams $1'$ and $2'$ have the same intensity.

\begin{figure}[!htb]
	\begin{center}
		\includegraphics[scale=1]{figures/ch_17/fig_17_16.pdf}
		\caption[]{}
        % \caption[]{(a) Scheme of the Michelson interferometer. Light travels from the source S and pass through the beam splitters P$_1$ and P$_2$, and is reflected by the mirrors M$_1$ and M$_2$. After multiple reflections, the rays are collected by the detector T. (b) The movable mirrors allow to change the thickness of the ``plate''; in particular, we can make planes M$_1$ and M$_2$ intersect.}
		\label{fig:17_16}
	\end{center}
	\vspace{-0.8cm}
\end{figure}

If conditions of temporal and spatial coherence are observed, beams $1'$ and $2'$ will interfere.
The result of this interference depends on the optical path difference from plate P$_1$ to mirrors M$_1$ and M$_2$, and back.
Ray $2$ passes through the plate three times, and ray $1$ only once.
To compensate the resulting change in the optical path difference (owing to dispersion) for waves of different lengths, plate P$_1$ is placed in the path of ray $1$.
Plates P$_1$ and P$_2$ are identical, except for the silver coating on the former.
This arrangement makes the paths of rays $1$ and $2$ in glass equal.
The interference pattern is observed with the aid of telescope T.

Let us mentally replace mirror M$_2$ with its virtual image M$_2'$ in semitransparent plate P$_1$.
Beams $1'$ and $2'$ can thus be considered as due to reflection from a transparent plate contained between planes M$_1$ and M$_2$.
We can use adjusting screws W$_1$ to change the angle between these planes; in particular, they can be arranged strictly parallel to each other.
By rotating micrometric screw W$_2$, we can smoothly move mirror M$_1$ without changing its inclination.
We can thus change the thickness of the ``plate''; in particular, we can make planes M$_1$ and M$_2$ intersect (\fig{17_16}b).

The nature of the interference pattern depends on the adjustment of the mirrors and on the divergence of the beam of light falling on the instrument.
If the beam is parallel, and planes M$_1$ and M$_2$ make an angle other than zero, then straight fringes of equal thickness parallel to the lines of intersection of planes M$_1$ and M$_2$ will be observed in the field of vision of the telescope.
In white light, all the fringes except the one coinciding with the line of intersection of the zero-order fringe will be coloured.
The zero-order fringe will be black because beam $1$ is reflected from plate P$_1$ from the outside, and beam $2$ from the inside.
As a result, a phase difference equal to $\pi$ is produced between them.
In white light, fringes are observed only with a small thickness of ``plate'' M$_1$M$_2'$ [see \eqn{17_36}].
In monochromatic light corresponding to the red line of cadmium, Michelson observed a distinct interference pattern at a path difference of the order of $500000$ wavelengths (the distance between M$_1$ and M$_2'$ in this case is about \SI{150}{\milli\metre}).

With a slightly diverging beam of light and a strictly parallel arrangement of planes M$_1$ and M$_2'$, fringes of equal inclination are obtained that have the form of concentric rings.
When micrometric screw W$_2$ is rotated, the diameter of the rings grows or diminishes.
Either new rings appear at the centre of the pattern, or the diminishing rings shrink to a point and then vanish.
Displacement of the pattern by one fringe corresponds to movement of mirror M$_2$ through half a wavelength.

Michelson used the instrument described above to carry out several experiments that entered the annals of physics.
The most famous of them, performed together with the American chemist Edward Morley (1838-1923) in 1887, had the aim of detecting motion of the Earth relative to the hypothetic ether (we shall treat this experiment in \sect{21_3}).
In 1890-1895, Michelson used the interferometer he had invented to make the first comparison of the wavelength of the red line of cadmium with the length of the standard metre.

In 1920, Michelson constructed a \textbf{stellar interferometer} which he used to measure the angular dimensions of stars.
This instrument was mounted on a telescope.
A screen with two slits was installed in front of the objective of the telescope (\fig{17_17}).
The light from a star was reflected from a symmetrical system of mirrors M$_1$, M$_2$, M$_3$ and M$_4$, installed on a rigid frame fastened on a carriage.
The inner mirrors M$_3$ and M$_4$, were fixed, and the outer ones M$_1$ and M$_2$, could move symmetrically away from or toward mirrors M$_3$ and M$_4$.
The path of the rays is clear from the figure.
Interference fringes were produced in the focal plane of the telescope objective.
Their visibility\footnote{The visibility of a fringe is defined as the quantity $V = (\ab{I}{max}-\ab{I}{min})/(\ab{I}{max}+\ab{I}{min})$, where $\ab{I}{max}$ and $\ab{I}{min}$ are the maximum and minimum intensities of the light in the vicinity of the given fringe, respectively.} depended on the distance between the outer mirrors.
By moving these mirrors, Michelson determined the distance $l$ between them at which the visibility of the fringes vanishes.
This distance must be of the order of the coherence radius of a light wave arriving from a star.
According to expression \eqref{eq:17_26}, the coherence radius is $l=\lambda/\varphi$.
The condition $l=\lambda/\varphi$ gives the angular diameter of a star
\begin{equation*}
    \varphi = \frac{\lambda}{l}.
\end{equation*}

\begin{figure}[!htb]
	\begin{center}
		\includegraphics[scale=0.9]{figures/ch_17/fig_17_17.pdf}
		\caption[]{}
        % \caption[]{Scheme of the stellar interferometer used by Michelson to measure the angular dimensions of stars.}
		\label{fig:17_17}
	\end{center}
	\vspace{-0.8cm}
\end{figure}

\noindent
Accurate calculations give the formula
\begin{equation*}
    \varphi = A \frac{\lambda}{l},
\end{equation*}

\noindent
where $A=1.22$ for a source in the form of a uniformly illuminated disk.
If the disk is darker at its edges than at the centre, the coefficient exceeds $1.22$, its value depending on the rate of diminishing of the illumination in the direction from the centre toward the edge.
In addition, accurate calculations show that after vanishing at a certain value of $l$, the visibility upon a further increase in $l$ again
becomes other than zero; however, the values it reaches are not great.

The maximum distance between the outer mirrors in the stellar interferometer constructed by Michelson was \SI{6.1}{\metre} (the diameter
of the telescope was \SI{2.5}{\metre}).
A minimum measurable angular diameter of about \ang{;0.02;} corresponded to this distance.
The first star whose angular diameter was measured was Betelgeuse (alpha Orion).
The value of $\varphi$ obtained for it was \ang{;0.047;}.

\section{Multibeam Interference}\label{sec:17_6}

Up to now, we have dealt with two-beam interference.
Now let us investigate the interference of many light rays.

Assume that $N$ rays of the same intensity arrive at a given point of a screen, the phase of each following ray being shifted relative
to that of the preceding one by the same value $\delta$.
Let us represent the oscillations set up by the rays in the form of exponents:
\begin{equation*}
    E_1 = a e^{i\omega t},\, E_2 = a e^{i(\omega t+\delta)}, \ldots, E_m = a e^{i[\omega t+(m-1)\delta]},\ldots, E_N = a e^{i[\omega t+(N-1)\delta]},
\end{equation*}

\noindent
where $a$ is the amplitude of an oscillation.
The resultant oscillation is determined by the formula
\begin{equation*}
    E = \sum_{m=1}^N E_m = a e^{i\omega t} \sum_{m=1}^N e^{i (m-1) \delta}.
\end{equation*}

\noindent
The expression obtained is the sum of $N$ terms of a geometrical progression with its first term equal to unity and its common ratio equal to $e^{i\delta}$.
Hence,
\begin{equation*}
    E = a e^{i\omega t} \parenthesis{\frac{1 - e^{iN\delta}}{1 - e^{i\delta}}} = \hat{A}\, e^{i\omega t},
\end{equation*}

\noindent
where
\begin{equation}\label{eq:17_43}
    \hat{A} = a \parenthesis{\frac{1 - e^{iN\delta}}{1 - e^{i\delta}}},
\end{equation}

\noindent
is the complex amplitude that can be represented in the form
\begin{equation}\label{eq:17_44}
    \hat{A} = A e^{i\alpha},
\end{equation}

\noindent
($A$ is the usual amplitude of the resultant oscillation, and $\alpha$ is its initial phase).

The product of quantity \eqref{eq:17_44} and its complex conjugate gives the square of the amplitude of the resultant oscillation:
\begin{equation}\label{eq:17_45}
    \hat{A}\hat{A}^* = A e^{i\alpha} A e^{-i\alpha} = A^2.
\end{equation}

\noindent
Substituting for $A$ in \eqn{17_45} its value from \eqn{17_43}, we get the following expression for the square of the amplitude:
\begin{align}
    A^2 &= \hat{A}\hat{A}^* = a^2 \frac{ \parenthesis{1 - e^{iN\delta}} \parenthesis{1 - e^{-iN\delta}} }{ \parenthesis{1 - e^{i\delta}} \parenthesis{1 - e^{-i\delta}}} = a^2 \frac{ \parenthesis{2 - e^{iN\delta} - e^{-iN\delta}} }{ \parenthesis{2 - e^{i\delta} - e^{-i\delta}} } \nonumber\\
    & = a^2 \bracket{ \frac{1 - \cos(N\delta)}{1 - \cos\delta} } = a^2 \frac{\sin^2(N\delta/2)}{\sin^2(\delta/2)}. \label{eq:17_46}
\end{align}

The intensity is proportional to the square of the amplitude.
Hence, the intensity produced upon the interference of the $N$ rays being considered is determined by the expression
\begin{equation}\label{eq:17_47}
    I(\delta) = K a^2 \frac{\sin^2(N\delta/2)}{\sin^2(\delta/2)} = I_0 \frac{\sin^2(N\delta/2)}{\sin^2(\delta/2)}
\end{equation}

\noindent
($K$ is a constant of proportionality, $I_0=Ka^2$ is the intensity produced by each of the rays separately).

At the values
\begin{equation}\label{eq:17_48}
    \delta = 2 \pi m \quad (m = 0, \pm 1, \pm 2, \ldots),
\end{equation}

\noindent
\eqn{17_47} becomes indeterminate.
For this reason, we apply L'Hospital's rule:
\begin{equation*}
    \lim_{\delta\to 2\pi m} \frac{\sin^2(N\delta/2)}{\sin^2(\delta/2)} = \lim_{\delta\to 2\pi m} \frac{ 2 \sin(N\delta/2) \cos(N\delta/2) (N/2)}{ 2 \sin(\delta/2) \cos(\delta/2) (1/2) } = \lim_{\delta\to 2\pi m} N \frac{ \sin(N\delta) }{\sin\delta}.
\end{equation*}

\noindent
The expression obtained is also indeterminate.
For this reason, we apply L'Hospital's rule again:
\begin{equation*}
    \lim_{\delta\to 2\pi m} \frac{\sin^2(N\delta/2)}{\sin^2(\delta/2)} = \lim_{\delta\to 2\pi m} N \frac{ \sin(N\delta) }{\sin\delta} = \lim_{\delta\to 2\pi m} N \frac{ \cos(N\delta) }{\cos\delta} = N^2.
\end{equation*}

Thus, when $\delta=2\pi m$ (or when the path differences $\Delta=m\lambda_0$), the resultant intensity is
\begin{equation}\label{eq:17_49}
    I = I_0 N^2.
\end{equation}

\noindent
This result could have been predicted.
Indeed, all the oscillations arrive at points for which $\delta=2\pi m$ in the same phase.
Hence, the resultant amplitude is $N$ times the amplitude of a separate oscillation, and the intensity is $N^2$ times that of a separate oscillation.

Let us call the spots where the intensity determined by \eqn{17_49} is observed the \textbf{principal maxima}.
Their position is determined by condition \eqref{eq:17_48}.
The number $m$ is called the \textbf{order} of the principal maximum.
It can be seen from \eqn{17_47} that the space between two adjacent principal maxima accommodates $N-1$ minima of the intensity.
To verify this statement, let us consider, for example, the interval between the maxima of the zero ($m=0$) and of the first ($m=1$) order.
In this interval, $\delta$ changes from zero to $2\pi$, and $\delta/2$ from zero to $\pi$.
The denominator of \eqn{17_47} is other than zero everywhere except for the ends of the interval.
It reaches its maximum value equal to unity at the middle of the interval.
The quantity $N\delta/2$ takes on all the values from zero to $N\pi$ within the interval being considered.
At values of $\pi, 2\pi, \ldots, (N-1)\pi$, the numerator of \eqn{17_47} becomes equal to zero.
Here, we have minima of the intensity.
Their positions correspond to values of $\delta$ equal to
\begin{equation}\label{eq:17_50}
    \delta = \frac{k'}{N} 2 \pi \quad (k' = 1, 2, \ldots, N-1).
\end{equation}

\noindent
There are $N-2$ secondary maxima in the intervals between the $N-1$ minima.
The secondary maxima closest to the principal maxima have the greatest intensity.
The secondary maximum closest to the principal zero-order maximum is between the first ($k'=1$) and second ($k'=2$) minima.
Values of $\delta$ equal to $2\pi/N$ and $4\pi/N$ correspond to these minima.
Hence, $\delta=3\pi/N$ corresponds to the secondary maximum being considered.
Introduction of this value into \eqn{17_47} yields
\begin{equation*}
    I(3\pi/N) = K a^2 \frac{ \sin^2(3\pi/N) }{ \sin^2(3\pi/2N) }.
\end{equation*}

\noindent
The numerator equals unity.
At a great value of $N$, we may assume that the sine in the denominator equals its argument [$\sin(3\pi/2N)\approx 3\pi/2N$].
Hence,
\begin{equation*}
    I(3\pi/N) = K a^2 \frac{1}{ (3\pi/2N)^2 } = \frac{K a^2 N^2}{(3\pi/2)^2}.
\end{equation*}

\noindent
The quantity in the numerator is the intensity of the principal maximum [see \eqn{17_49}].
Thus, at a great value of $N$, the secondary maximum closest to the principal maximum has an intensity that is $1/(3n/2)^2\approx 1/22$ of the intensity of the principal maximum.
The other secondary maxima are still weaker.

\begin{figure}[!htb]
	\begin{center}
		\includegraphics[scale=0.95]{figures/ch_17/fig_17_18.pdf}
		\caption[]{}
        % \caption[]{Plot of the function $I(\delta)$ for $N=10$. The principal maxima become narrower and narrower with an increase in the number of interfering rays. The secondary maxima are so weak that the interference pattern practically has the form of narrow bright lines on a dark background.}
		\label{fig:17_18}
	\end{center}
	\vspace{-0.85cm}
\end{figure}

Figure \ref{fig:17_18} shows a plot of the function $I(\delta)$ for $N=10$.
For comparison, a plot of the intensity for $N=2$ [two-beam interference; see the curve $I(x)$ in \fig{17_2}] is shown by a dash line.
Inspection of the figure shows that the principal maxima become narrower and narrower with an increase in the number of interfering rays.
The secondary maxima are so weak that the interference pattern practically has the form of narrow bright lines on a dark background.

Now, let us consider the interference of a very great number of rays whose intensity diminishes in a geometrical progression.
The oscillations being added have the form
\begin{equation}\label{eq:17_51}
    E_1 = a e^{i\omega t},\, E_2 = a \rho e^{i(\omega t+\delta)},\ldots, E_m = a \rho^{m-1} e^{i[\omega t+(m-1)\delta]}, \ldots,
\end{equation}

\noindent
($\rho$ is a constant quantity less than unity).
The resultant oscillation is described by the equation
\begin{equation*}
    E = \sum_{m=1}^N E_m = a e^{i\omega t} \sum_{m-1}^N \rho^{m-1} e^{i(m-1)\delta}.
\end{equation*}

\noindent
Using the expression for the sum of the terms of a geometrical progression, we get
\begin{equation*}
    E = a e^{i\omega t} \parenthesis{ \frac{1 - \rho e^{iN\delta}}{1 - \rho e^{i\delta}} } = \hat{A} e^{i\omega t}.
\end{equation*}

\noindent
Thus, the complex amplitude is
\begin{equation}\label{eq:17_52}
    \hat{A} = a \parenthesis{ \frac{1 - \rho e^{iN\delta}}{1 - \rho e^{i\delta}} }.
\end{equation}

If $N$ is very great, the complex number $\rho Ne^{iN\delta}$ may be disregarded in comparison with unity (we shall indicate as an example that $0.9^{100}\approx\num{4e-4}$).
Equation \eqref{eq:17_52} is thus simplified as follows:
\begin{equation*}
    \hat{A} = a \parenthesis{ \frac{1}{1 - \rho e^{i\delta}} }.
\end{equation*}

\noindent
Multiplying this equation by its complex conjugate, we get the square of the ordinary amplitude of the resultant oscillation:
\begin{align*}
    A^2 &= \hat{A}\hat{A}^* = \frac{a^2}{ \parenthesis{1 - \rho e^{i\delta}} \parenthesis{1 - \rho e^{-i\delta}} } = \frac{a^2}{ 1 + \rho^2 - \rho \parenthesis{e^{i\delta} + e^{-i\delta}} }\\
    &= \frac{a^2}{ 1 + \rho^2 - 2 \rho \cos\delta } = \frac{a^2}{ (1-\rho)^2 + 2\rho (1-\cos\delta) } \\
    &= \frac{a^2}{ (1-\rho^2) + 4\rho\sin^2(\delta/2) }.
\end{align*}

\noindent
Hence,
\begin{equation}\label{eq:17_53}
    I(\delta) = \frac{Ka^2}{(1-\rho^2) + 4\rho\sin^2(\delta/2)} = \frac{I_1}{(1-\rho^2) + 4\rho\sin^2(\delta/2)},
\end{equation}

\noindent
where $I_1=Ka^2$ is the intensity of the first (most intensive) ray.

At values of
\begin{equation}\label{eq:17_54}
    \delta = 2 \pi m \quad (m = 0, \pm 1, \pm 2, \ldots),
\end{equation}

\noindent
\eqn{17_53} has maxima equal to
\begin{equation}\label{eq:17_55}
    \ab{I}{max} = \frac{I_1}{(1-\rho)^2}.
\end{equation}

\noindent
In the intervals between maxima, the function changes monotonously, reaching a value equal to
\begin{equation}\label{eq:17_56}
    \ab{I}{min} = \frac{I_1}{(1-\rho)^2 + 4\rho} = \frac{I_1}{(1+\rho)^2}
\end{equation}

\noindent
at the middle of the interval.
Thus, the ratio of the intensity at a maximum to that at a minimum
\begin{equation}\label{eq:17_57}
    \frac{\ab{I}{max}}{\ab{I}{min}} = \parenthesis{\frac{1+\rho}{1-\rho}}^2
\end{equation}

\noindent
is the greater, the closer $\rho$ is to unity, \ie, the slower is the rate of diminishing of the intensity of the interfering rays.
Figure \ref{fig:17_19} shows a graph of function \eqref{eq:17_53} for $\rho=0.8$.
It can be seen from the figure that the interference pattern has the form of narrow sharp
lines on a virtually dark background.
Unlike \fig{17_18}, secondary maxima are absent.

\begin{figure}[!htb]
	\begin{center}
		\includegraphics[scale=0.95]{figures/ch_17/fig_17_19.pdf}
		\caption[]{}
        % \caption[]{Graph of function \eqref{eq:17_53} for $\rho=0.8$. The interference pattern has the form of narrow sharp lines on a virtually dark background. Secondary maxima are absent.}
		\label{fig:17_19}
	\end{center}
	\vspace{-0.85cm}
\end{figure}

A practical case of a great number of rays with a diminishing intensity is encountered in the \textbf{Fabry-Perot interferometer}.
This instrument consists of two glass or quartz plates separated by an air gap (\fig{17_20}).
The internal surfaces of the plates are thoroughly polished so that the irregularities on them do not exceed several hundredths of the length of a light wave.
Next partly transparent metal layers or dielectric films\footnote{Metal layers have the shortcoming that they absorb light rays to a great extent. This is why recent years have seen their replacement with multilayer dielectric coatings having a high reflectivity.} are applied to these surfaces.
The outer surfaces of the plates are at a slight angle relative to the inner ones to eliminate the highlights due to the reflection of light from these surfaces.
In the original design of the interferometer, one of the plates could be moved relative to the other stationary one with the aid of a micrometric screw.
The unreliability of this design, however, resulted in its coming out of use.
In modern designs, the plates are secured rigidly.
The parallelity of the internal working planes is achieved by installing an invar or quartz ring\footnote{Both these materials are distinguished by their extremely low temperature
coefficient of expansion.} between the plates.
This ring has three projections with thoroughly polished edges at each side.
The plates are pressed against the ring by springs.
This design reliably ensures strict parallelity of the internal planes of the plates and constancy of the distance between them.
Such an interferometer with a fixed distance between its plates is known as a \textbf{Fabry-Perot etalon}.

\begin{figure}[!htb]
	\begin{minipage}[t]{0.38\linewidth}
		\begin{center}
			\includegraphics[scale=1]{figures/ch_17/fig_17_20.pdf}
			\caption[]{}
            % \caption[]{Scheme of a Fabry-Perot interferometer. It consists of two glass or quartz plates separated by an air gap. The internal surfaces of the plates are thoroughly polished so that the irregularities on them do not exceed several hundredths of the length of a light wave.}
			\label{fig:17_20}
		\end{center}
	\end{minipage}
	\hfill{ }%space{-0.05cm}
	\begin{minipage}[t]{0.62\linewidth}
		\begin{center}
			\includegraphics[scale=0.85]{figures/ch_17/fig_17_21.pdf}
            \caption[]{}
			% \caption[]{Ray entering the gap between the plates as described by \fig{17_20}. Interference in a Fabry-Perot interferometer.}
			\label{fig:17_21}
		\end{center}
	\end{minipage}
\vspace{-0.4cm}
\end{figure}

Let us see what happens to a ray entering the gap between the plates (\fig{17_21}).
Assume that the intensity of the entering ray is $I_0$.
At point A$_1$, this ray is divided into ray $1$ emerging outward and reflected ray $1'$.
If the coefficient of reflection from the surface of the plate is $\rho$, then the intensity of ray $1$ will be $I_1=(1-p)I_0$, and the intensity of the reflected ray will be $I_1'=\rho I_0$\footnote{We disregard the absorption of light in the reflecting layers and inside the plates.}.
At point B$_1$, ray $1'$ is divided into two.
Ray $1''$ shown by a dash line will drop out of consideration, while reflected ray $1''$ will have an intensity of $I_1''=\rho I_1'=\rho^2I_0$.
At point A$_2$, ray $1''$ will be divided into two rays---ray $2$ emerging outward having an intensity of $I_2=(1-\rho)I_1''=(1-\rho)\rho^2/I_0$ and reflected ray $2'$, and so on.
Thus, the following relation holds for the intensities of rays $1$, $2$, $3$, etc. emerging
from the instrument:
\begin{equation*}
    I_1 : I_2 : I_3 : \ldots = 1 : \rho^2 : \rho^4 : \ldots .
\end{equation*}

\noindent
Accordingly, for the amplitudes of the oscillations we have
\begin{equation*}
    A_1 : A_2 : A_3 : \ldots = 1 : \rho : \rho^2 : \ldots
\end{equation*}

\noindent
[compare with \eqn{17_51}].

The oscillation in each of the rays $2$, $3$, $4$, $\ldots$, lags in phase behind the oscillation in the preceding ray by the same amount $\delta$ determined by the optical path difference $\Delta$ appearing on the path A$_1$-B$_1$-A$_2$ or A$_2$-B$_2$-A$_3$, etc. (see \fig{17_21}).
A glance at the figure shows that $\Delta=2l/\cos\varphi$, where $\varphi$ is the angle of incidence of the rays on the reflecting layers.

If we gather rays $1$, $2$, $3$, $\ldots$, with the aid of a lens at point P of its focal plane (see \fig{17_20}), then the oscillations produced by these rays will have the form given by \eqn{17_51}.
Hence, the intensity at point P is determined by \eqn{17_53}, in which $\rho$ has the meaning of the coefficient of reflection, and
\begin{equation*}
    \delta = \frac{2\pi}{\lambda} = \frac{2l}{\cos\varphi}.
\end{equation*}

\begin{figure}[t]
	\begin{center}
		\includegraphics[scale=0.9]{figures/ch_17/fig_17_22.pdf}
		\caption[]{}
        % \caption[]{When a diverging beam of light is passed through the Fabry-Perot interferometer instrument, fringes of equal inclination having the form of sharp rings are produced in the focal plane of the lens.}
		\label{fig:17_22}
	\end{center}
	\vspace{-0.8cm}
\end{figure}

When a diverging beam of light is passed through the instrument, fringes of equal inclination having the form of sharp rings (\fig{17_22}) will be produced in the focal plane of the lens.

The Fabry-Perot interferometer is used in spectroscopy to study the fine structure of spectral lines.
It has also come into great favour in metrology for comparing the length of the standard metre with the wavelengths of individual spectral lines.