% !TEX root = saveliev_physics_general_course_1.tex
%!TEX TS-program = pdflatex
%!TEX encoding = UTF-8 Unicode


\chapter*{AUTHOR'S PREFACE TO\\ THE ENGLISH EDITION}
\addcontentsline{toc}{chapter}{Preface}
\chaptermark{\sc Preface}

%\vspace{-1cm}

%\noindent
The present book is the first volume of the three-volume general course in physics. The course is a result of twenty five year's work in the Department of General Physics of the Moscow Institute of Engineering Physics. I was in constant personal contact with my students not only at lectures, but also, perhaps even more importantly, at exercises, consultations, and examinations. These fruitful contacts helped me refine and improve the exposition of the various topics in the course.

The advice and friendly criticism of my colleagues in the department has also been a great help. I would like to make a special mention of the part played by N.~B.~Narozhny, who, in particular, is the author of the original and comparatively simple statistical derivation of the equation $\mathrm{d}S = \mathrm{d}'Q/T$ [\eqn{11_110}].

In writing the book, I have done everything in my power to acquaint students with the basic ideas and methods in physics and to teach them how to think physically. This is why the book is not encyclopedic in its nature. It is mainly devoted to explaining the meaning of physical laws and showing how to apply them consciously. What I have tried to achieve is a deep knowledge of the fundamental principles of physics rather that a shallower acquaintance with the a wide range of questions.

While using the book, try not to memorise the material formalistically and mechanically, but logically, \textit{i.e.}, memorise the material by thoroughly understanding it. I have tried to present physics not as a science for  ``cramming'', not as a certain volume of information to be memorised, but as a clever, logical, and attractive science. It is left to the reader to judge the extent to which I have succeeded in doing this.

Acknowledging the fact that a thick volume by its very appearance makes a student despondent, I have done my utmost to limit the size of the course. This was achieved by carefully choosing the material which in my opinion should be included in a general course of physics. I also tried to be concise, but not at the expense of clarity.

Notwithstanding my desire to reduce the size, I considered it essential to include a number of mathematical sections in the course: on vectors, linear differential equations, the basic concepts of the theory of probability, etc. This was done to impart a ``physical'' tinge to the relevant concepts and relations. In addition, the mathematical ``inclusions'' make it possible to go on with the physics even if, as is often the case, the relevant material has not yet been covered in a mathematics course.

The present course is intended above all for higher technical schools with an extended syllabus in physics. The material has been arranged, however, so that the book can be used as a teaching aid for higher technical schools with ordinary syllabus simply omitting some sections.

\begin{flushright}
	\emph{Igor Savelyev}
\end{flushright}

\noindent
Moscow, July, 1979

%\pagestyle{mystyle}
